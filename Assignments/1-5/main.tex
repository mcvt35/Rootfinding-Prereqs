\documentclass[12pt,oneside]{article}

% This package simply sets the margins to be 1 inch.
\usepackage[margin=1in]{geometry}

% These packages include nice commands from AMS-LaTeX
\usepackage{amssymb,amsmath,amsthm,graphicx}


% Define an environment for exercises.
\newenvironment{exercise}[1]{\vspace{.1in}\noindent\textbf{Exercise #1 \hspace{.05em}}}{}

% define shortcut commands for commonly used symbols
\newcommand{\R}{\mathbb{R}}
\newcommand{\C}{\mathbb{C}}
\newcommand{\Z}{\mathbb{Z}}
\newcommand{\Q}{\mathbb{Q}}
\newcommand{\N}{\mathbb{N}}


%%%%%%%%%%%%%%%%%%%%%%%%%%%%%%%%%%%%%%%%%%

\begin{document}

% If you use Overleaf, the name of the project will be determined by
% what you enter as the document title.
\title{Math Homework Template}

\begin{flushright}
\textsc{Marcelo Leszynski}  \\
Rootfinding Research Prerequisites\\
07/13/20
\end{flushright}

\begin{center}
\textsf{Assignment 1.5 } \\
\textsf{Exercises: 1, 2, 3, 4, 5, 6, 7, 10, 11, 12 }
\end{center}

%%%%%%%%%%%%%%%%%%%%%%%%%%%%%%%%%%%%%%%%

\begin{exercise}{1.5.1}
    \begin{proof}
        Let $f \in \C[x]$ be a polynomial of degree $n > 0$. We wish to 
        show that $f$ can be written in the form $f=c(x-a_1)\ldots(x-a_n)$,
        where $c,a_1,\ldots,a_n \in \C$ and $c \neq 0$.

        We begin by noting that f has some root $r_1 \in \C$ by Theorem 1.1.7.
        This allows us to rewrite $f = f_1(x-a_1)$ for some $f_1 \in \C[x]$. 
        By Corollary 1.5.3, we know that $f_1$ has a degree of up to $n-1$. We 
        can proceed similarly by acknowledging that $f_1$ has a root $r_2 \in \C$
        such that $f_1 = f_2(x-r_2)$ for some $f_2 \in \C[x]$ of degree $n-2$. We 
        repeat this process $n$ times to get $f_1,\ldots,f_{n-1}$ with 
        $f_{n-1} = cx+d$ having degree 1. Then $c \neq 0$, so $f_{n-1} = c(x-r_n)$
        for $r_n = -d/c$. We now have the following
        \begin{align*}
            f   &= f_1(x-r_1) = f_2(x-r_2)(x-r_1) = \ldots \\
                &= f_{n-1}(x-r_{n-1})\ldots(x-r_1) \\
                &= c(x-r_n)(x-r_{n-1})\ldots(x-r_1) \\
                &= c(x-r_1)\ldots(x-r_n)
        \end{align*}
        as desired.
    \end{proof}
\end{exercise}

%%%%%%%%%%%%%%%%%%%%%%%%%%%%%%%%%%%%%%%%

\begin{exercise}{1.5.2}
    \begin{proof}
        Let $A$ be the matrix pictured in the problem. If we suppose that 
        the determinant $\det(A) = 0$, then there exists some vector $\vec{v} \in k^n$
        such that $A\vec{v} = \vec{0}$ with $\vec{v} \neq \vec{0}$.

        Let $\vec{v} = \langle c_0, \ldots ,c_{n-1} \rangle^T$. Next, define a polynomial
        that represents a row of $A$ as $p(x) = c_{n-1}x^{n-1} + \ldots + c_0$. 
        Since the degree of $p(x)$ is at most $n-1$, it can have up to $n-1$ 
        distinct roots. Observe that $\det(A)=0$ implies that $p(a_i) = c_{n-1}a_i^{n-1}+\ldots+c_0=0$
        for all $1 \leq i \leq n, i \in \N$. Thus we have $n$ distinct roots resulting 
        from distinct $a_i$ values, which contradicts our previous finding. Therefore,
        we can conclude by contradiction that $\det(A) \neq 0$.
    \end{proof}
\end{exercise}

%%%%%%%%%%%%%%%%%%%%%%%%%%%%%%%%%%%%%%%%

\begin{exercise}{1.5.3}
    \begin{proof}
        We wish to show that $I = \langle x,y \rangle \subseteq k[x,y]$ is not a 
        principal ideal. In other words, $I$ cannot be generated by one element. 
        We proceed with a proof by contradiction.

        Suppose, by way of contradiction, that $\langle x,y \rangle = \langle g \rangle$
        for some $g \in k[x,y]$. Then $g$ divides $x$, or in other words, 
        $x=fg$ for some $f \in k[x,y]$. Rewriting $f = \sum_if_i(y)x^i$ and 
        $g = \sum_jg_j(y)x^j$, we have that 
        \[
            x = fg = \biggr(\sum_if_i(y)x^i\biggr)\biggr(\sum_jg_j(y)x^j\biggr)=\sum_t\biggr(\sum_{i+j=t}f_i(y)g_j(y)\biggr)x^t.
        \]
        There are only two cases where this is possible.

        \underline{Case 1}: $f=c$ and $g=dx$ with $c,d \in k$ satisfying $cd = 1$.
        This implies that $y \in \langle x,y \rangle = \langle dx \rangle$ is divisible by 
        $x$, which is impossible.

        \underline{Case 2}: $f=cx$ and $g=d$ with $c,d \in k$ satisfying $cd = 1$. 
        This implies that $\langle x,y \rangle = \langle d \rangle$. This is impossible 
        since $1 \not \in \langle x,y \rangle$.

    \end{proof}
\end{exercise}

%%%%%%%%%%%%%%%%%%%%%%%%%%%%%%%%%%%%%%%%

\begin{exercise}{1.5.4}
    \begin{proof}
        Assume that $h = \gcd(f,g)$. Then, by Proposition 1.5.6, $h$ is a
        generator of $\langle f,g \rangle$. That is, $\langle h \rangle = \langle f,g \rangle$.
        Note that $h = 1 \cdot h \in \langle h \rangle = \langle f,g \rangle$, which 
        implies that $h = Af + Bg$ for some $A,B \in k[x]$ by definition of the ideal
        $\langle f,g \rangle$. 
    \end{proof}
\end{exercise}

%%%%%%%%%%%%%%%%%%%%%%%%%%%%%%%%%%%%%%%%

\begin{exercise}{1.5.5}
    \begin{proof}
        Let $f,g \in k[x]$. We wish to show that $\langle f-qg,g \rangle = \langle f,g \rangle$
        for any $q \in k[x]$.

        \bigskip
        \textbf{($\subseteq$):} 
        \begin{align*}
            f-qg    &= 1 \cdot f + (-q) \cdot g \in \langle f,g \rangle\\
            g       &= 0 \cdot f + 1 \cdot g \in \langle f,g \rangle    
        \end{align*}
        so $\langle f-qg,g \rangle \subseteq \langle f,g \rangle$.

        \bigskip
        \textbf{($\supseteq$):} 
        \begin{align*}
            f   &= 1 \cdot (f-qg) + q \cdot g \in \langle f-qg, g \rangle\\
            g   &= 0 \cdot (f-qg) + 1 \cdot g \in \langle f-qg, g \rangle
        \end{align*}
        so $\langle f,g \rangle \subseteq \langle f-qg, g \rangle$.

        Therefore, $\langle f-qg,g \rangle = \langle f,g \rangle$ as desired.
    \end{proof}
\end{exercise}

%%%%%%%%%%%%%%%%%%%%%%%%%%%%%%%%%%%%%%%%

\begin{exercise}{1.5.6}
    \begin{proof}
        Let $f_1,\ldots,f_s \in k[x]$ and let $h=\gcd(f_2,\ldots,f_s)$. We 
        wish to show that $\langle f_1,h \rangle = \langle f_1,f_2,\ldots,f_s \rangle$.

        \bigskip
        \textbf{($\subseteq$):} By Proposition 1.5.6, we know that $h$ is a generator of 
        $\langle f_2,\ldots,f_s \rangle$, i.e.,
        $\langle h \rangle = \langle f_2,\ldots,f_s \rangle \subseteq \langle f_1,f_2,\ldots,f_s \rangle$.
        We observe that $f_1 \in \langle f_1,f_2,\ldots,f_s \rangle$. Thus 
        $\langle f_1, h \rangle \subseteq \langle f_1,f_2,\ldots,f_s \rangle$.

        \bigskip
        \textbf{($\supseteq$):} Note that $f_1 \in \langle f_1,h \rangle$ and that,
        for $2 \leq i \leq s$, $f_i \in \langle f_2,\ldots,f_s \rangle = \langle h \rangle \subseteq \langle f_1, h \rangle$.
        Thus $\langle f_1,f_2,\ldots,f_s \rangle \subseteq \langle f_1,h \rangle$.

        Therefore, $\langle f_1,h \rangle = \langle f_1,f_2,\ldots,f_s \rangle$.
    \end{proof}
\end{exercise}

%%%%%%%%%%%%%%%%%%%%%%%%%%%%%%%%%%%%%%%%

\begin{exercise}{1.5.7}
    The algorithm is as follows:
    \begin{align*}
        &\text{Input: }f_1,\ldots,f_s \in k[x], s\geq 2\\
        &\text{Output: }h = \gcd(f_1,\ldots,f_s)\\
        \\
        &h:=f_s\\
        &\text{FOR }i=s-1 \text{ TO } 1 \text{ DO \{}\\
        &h:=\gcd(f_i,h)\\
        &\}\\
        &\text{RETURN }h
    \end{align*}
\end{exercise}

%%%%%%%%%%%%%%%%%%%%%%%%%%%%%%%%%%%%%%%%

\begin{exercise}{1.5.10}
    The algorithm is as follows:
    \begin{align*}
        &\text{Input: }f,g \in k[x]\\
        &\text{Output: }h=\gcd(f,g),A,B \in k[x] \text{ with } Af+Bg=h\\
        \\
        &h:=f\\
        &s:=g\\
        &A:=1\\
        &B:=0\\
        &C:=0\\
        &D:=1\\
        &\text{WHILE }s \neq 0 \text{ DO \{}\\
        &r:= \text{remainder}(h,s)\\
        &q:= \text{quotient}(h,s)\\
        &h:=s\\
        &s:=r\\
        &\text{TempA}:=A\\
        &\text{TempB}:=B\\
        &A:=C\\
        &B:=D\\
        &C:=\text{ TempA}-q*C\\
        &D:=\text{ TempB}-q*D\\
        &\}\\
        &\text{RETURN }h,A,B
    \end{align*}
\end{exercise}

%%%%%%%%%%%%%%%%%%%%%%%%%%%%%%%%%%%%%%%%

\newpage
\begin{exercise}{1.5.11}
    
    \bigskip
    \textbf{(1.5.11a):} Let $f \in \C[x]$ be nonzero. We wish to show that 
    $V(f) = \emptyset$ if and only if $f$ is constant and will proceed by 
    proving the contrapositive statements
    $V(f) \neq \emptyset \Leftrightarrow f$ is nonconstant.

    \bigskip
    ($\Rightarrow$): Assume $V(f) \neq \emptyset$, so there exists some 
    $a \in V(f)$. This implies that $f(a) = 0$ so $f$ must be nonconstant 
    since we assumed $f$ to be nonzero.

    \bigskip
    ($\Leftarrow$): Assume $f$ is nonconstant. Then, by Theorem 1.1.7, there
    exists some root $a \in \C$ which implies $a \in V(f)$ so $V(f) \neq \emptyset$.

    Therefore, $V(f) = \emptyset$ if and only if $f$ is constant.

    \bigskip
    \textbf{(1.5.11b):} Let $f_1,\ldots,f_s \in \C[x]$. We wish to show that 
    \[
        V(f_1,\ldots,f_s) = \emptyset \Leftrightarrow \gcd(f_1,\ldots,f_s) = 1.
    \]

    \bigskip
    ($\Rightarrow$): Let $f = \gcd(f_1,\ldots,f_s)$. Then $f$ is a generator for 
    $\langle f_1,\ldots,f_s \rangle$ such that $\langle f \rangle = \langle f_1,\ldots,f_s \rangle$.
    Proposition 1.4.4 then gives that $V(f_1,\ldots,f_s) = V(f) = \emptyset$. This 
    implies that $f$ is a constant, and it follows that $f=1$ since any other 
    nonzero value implies that $V(f_1,\ldots,f_2) \neq \emptyset$.

    \bigskip
    ($\Leftarrow$): Let $f = \gcd(f_1,\ldots,f_s) = 1$. It follows that $f$ 
    is a generator for $\langle f_1,\ldots,f_s \rangle$, i.e., $\langle f \rangle = \langle f_1,\ldots f_s \rangle$.
    Note that $V(f) = \emptyset$ since $f$ is a constant.
    Proposition 1.4.4 then gives that $V(f) = V(f_1,\ldots,f_s) = \emptyset$.

    Therefore, $V(f_1,\ldots,f_s) = \emptyset$ if and only if $\gcd(f_1,\ldots,f_s) = 1$.

    \bigskip
    \textbf{(1.5.11c):} Given an arbitrary set of polynomials $f_1,\ldots,f_s \in \C[x]$, 
    compute the gcd $f=\gcd(f_1,\ldots,f_s)$. If $f=1$, then $V(f_1,\ldots,f_s) = \emptyset$.
    If $f \neq 1$, then $V(f_1,\ldots,f_s) \neq \emptyset$. This is true because 
    of the biconditional that was proven in Exercise 1.5.11b.
\end{exercise}

%%%%%%%%%%%%%%%%%%%%%%%%%%%%%%%%%%%%%%%%

\begin{exercise}{1.5.12}

    \bigskip
    \textbf{(1.5.12a):} $V(f) = \{a_1,\ldots,a_l\}$ follows by definition of $f = c(x-a_1)^{r_1}\ldots (x-a_l)^{r_l}$.

    \bigskip
    \textbf{(1.5.12b):} Let $f_{red} = c(x-a_1)\ldots(x-a_l)$. We wish to show 
    that $I(V(f)) = \langle f_{red} \rangle$.

    \bigskip
    ($\subseteq$): Let $g \in I(V(f))$, i.e., $g$ vanishes at $\{a_1,\ldots,a_l\}$.
    This means that $g$ has at least $l$ roots labeled $a_1,\ldots,a_l,a_{l+1},\ldots,a_m$
    where $m \geq l$. Since we are over $\C$, we have that 
    \[
        g=d(x-a_1)^{s_1}\ldots(x-a_l)^{s_l}(x-a_{l+1})^{s_{l+1}}\ldots(x-a_m)^{s_m}
    \]
    with $d \in \C$ such that $d \neq 0$ and $s_i \geq 1$ for all $1\leq i \leq m$.
    Hence, $g$ is a multiple of $(x-a_1)\ldots(x-a_l)$ and thus it is a multiple 
    of $f_{red} = c(x-a_1)\ldots(x-a_l)$ (since $c \neq 0$). Thus $g \in \langle f_{red} \rangle$.
    Since $g$ is arbitrary, we have shown that $I(V(f)) \subseteq \langle f_{red} \rangle$.
   
    \bigskip
    ($\supseteq$): Note that $f_{red}$ 
    vanishes on $\{a_1,\ldots,a_l\} = V(f)$. Thus $f_{red} \in I(V(f))$ and 
    so $\langle f_{red} \rangle \subseteq I(V(f))$. 

    Therefore, we conclude that $I(V(f)) = \langle f_{red} \rangle$ as desired.
\end{exercise}

%%%%%%%%%%%%%%%%%%%%%%%%%%%%%%%%%%%%%%%%

%---------------------------------
% Don't change anything below here
%---------------------------------


\end{document}