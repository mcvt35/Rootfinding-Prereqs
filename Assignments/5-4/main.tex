\documentclass[12pt,oneside]{article}

% This package simply sets the margins to be 1 inch.
\usepackage[margin=1in]{geometry}

% These packages include nice commands from AMS-LaTeX
\usepackage{amssymb,amsmath,amsthm,graphicx}

% Define an environment for exercises.
\newenvironment{exercise}[1]{\vspace{.1in}\noindent\textbf{Exercise #1 \hspace{.05em}}}{}

% define shortcut commands for commonly used symbols
\newcommand{\R}{\mathbb{R}}
\newcommand{\C}{\mathbb{C}}
\newcommand{\Z}{\mathbb{Z}}
\newcommand{\Q}{\mathbb{Q}} 
\newcommand{\N}{\mathbb{N}} 

%%%%%%%%%%%%%%%%%%%%%%%%%%%%%%%%%%%%%%%%%%

\begin{document}

% If you use Overleaf, the name of the project will be determined by
% what you enter as the document title.
\title{Math Homework Template}

\begin{flushright}
\textsc{Marcelo Leszynski}  \\
Rootfinding Research Prerequisites\\
10/07/20
\end{flushright}

\begin{center}
\textsf{Assignment 5.4 } \\
\textsf{Exercises: 2, 4, 5, 8, 11 }
\end{center}

%%%%%%%%%%%%%%%%%%%%%%%%%%%%%%%%%%%%%%%%

\begin{exercise}{5.4.2}
    
    \bigskip
    \textbf{(5.4.2a):}
    Let $V$ be defined as in the text, and let $\phi=[f], f\in \C[x_1,\ldots,x_n]$.
    It follows that 
    \[
        \mathbf{V}_V(\phi)=\{(x_1,\ldots,x_n) \in V \vert \phi(x_1,\ldots,x_n)=0\}=\mathbf{V}(f_1,\ldots,f_s,f)
    \]
    for $f_1,\ldots,f_s \in V$. We then have 
    \begin{align*}
        \mathbf{V}_V(\phi)=0    &\Leftrightarrow \langle f_1,\ldots,f_s,f\rangle = \C[x_1,\ldots,x_n] && \text{Weak Nullstellensatz}\\
                                &\Leftrightarrow h_1f_1+\ldots+h_sf_s+hf=1&&\text{for some }h_1,\ldots,h_s,h \in \C[x_1,\ldots,x_n]\\
                                &\Leftrightarrow [h]\phi = [1] \in \C[V]&&\text{since }[h_1f_1+\ldots+h_sf_s]=[0] \in \C[V]\\
                                &\Leftrightarrow \phi \text{ is invertible in }\C[V].
    \end{align*}
    
    \bigskip
    \textbf{(5.4.2b):}
    The statement above is not true over $\R$ as can be seen by the 
    counterexample $V=\mathbf{V}(y) \subseteq \R^2, \phi=x^2+1$. It 
    follows that $\mathbf{V}_V(\phi)=\emptyset$ since $\phi$ has no root in 
    $\R^2$. 
    
    It is also true that $\phi$ is not invertible in $\R[V]$. 
    Supposing that this is not the case, there exists some $\psi=[g] \in \R[V]$ 
    such that $\psi\phi=[1]$. By proposition 5.1.2, this implies that 
    $(x^2+1)\cdot g(x,y)-1 = h(x,y) \cdot y$ for some $h(x,y)\in \R[x,y]$. 
    This is equivalent to saying that $(x^2+1)\cdot g(x,0)=1$, which is 
    impossible since $g(x,0)$ is a polynomial in $\R[x]$. Therefore, 
    $\phi$ is not invertible in $\R[V]$.

\end{exercise}

%%%%%%%%%%%%%%%%%%%%%%%%%%%%%%%%%%%%%%%%

\begin{exercise}{5.4.4}
    Let $V$ be as defined in the text, and define the mappings 
    $\alpha:k\to V, \beta:V\to k$ as $\alpha(x)=(x,x^n,x^m)$ and 
    $\beta(x,y,z)=x$, respectively. We take their compositions and
    have that 
    \begin{align*}
        (\alpha \circ \beta)(x,y,z)&=\alpha(x)=(x,x^n,x^m)=(x,y,z), (x,y,z)\in V,\\
        (\beta \circ \alpha)(x) &= \beta(x,x^n,x^m)=x.
    \end{align*}
    Therefore $\alpha \circ \beta = id_V$ and $\beta \circ \alpha = id_k$,
    so $V$ is isomorphic as a variety to $k$.
\end{exercise}

%%%%%%%%%%%%%%%%%%%%%%%%%%%%%%%%%%%%%%%%

\begin{exercise}{5.4.5}
    The processes are similar enough that we demonstrate only 
    that the surface $V \in k^3$ defined by $x-f(y,z)=0$ is 
    isomorphic as a variety to $k^2$. We do so by defining the 
    polynomial mappings $\alpha:k^2\to V$ and $\beta:V\to k^2$ 
    as $\alpha(y,z)=(f(y,z)y,z)$ and $\beta(x,y,z)=(y,z)$, respectively.
    Taking the compositions yields
    \begin{align*}
        (\alpha \circ \beta)(x,y,z) &= \alpha(y,z)=(f(y,z),y,z)=(x,y,z), (x,y,z)\in V\\
        (\beta \circ \alpha)(y,z) &= \beta(f(y,z),y,z)=(y,z).
    \end{align*}
    Therefore $\alpha \circ \beta = id_V$ and $\beta \circ \alpha = id_{k^2}$,
    so it follows that $V$ is isomorphic as a variety to $k^2$.
\end{exercise}

%%%%%%%%%%%%%%%%%%%%%%%%%%%%%%%%%%%%%%%%

\begin{exercise}{5.4.8}
    
    \bigskip
    \textbf{(5.4.8a):}
    Let $Q_1$ and $Q_2$ be as in the text. The pencil of surfaces 
    determined by $Q_1$ and $Q_2$ is given by 
    \[
        \{Q_2\}\cup\{F_c=\mathbf{V}(f_1+cf_2)\vert c\in R\}.    
    \]
    Fixing $c=-1$ constrains $F_{-1}$ to be 
    \[
        0=(x^2+y^2+z^2-1)-(x^2-x+\frac{1}{4}-3y^2-2z^2) = x+4y^2+3z^3-\frac{5}{4}   
    \]
    Using Exercise 5.4.5, the surface $Q=F_{-1}=\mathbf{V}(x+4y^2+3z^3-\frac{5}{4})$
    is isomorphic as a variety to $\R^2$.
    
    \bigskip
    \textbf{(5.4.8b):}
    $Q_1$ is the unit sphere and $Q_2$ is a cone with its vertex 
    at $(\frac{1}{2},0,0)$. Thus, their intersections are two 
    ellipses in planes parallel to the $(y,z)$-plane.

\end{exercise}

%%%%%%%%%%%%%%%%%%%%%%%%%%%%%%%%%%%%%%%%

\begin{exercise}{5.4.11}
    
    \bigskip
    \textbf{(5.4.11a):}
    To begin, we show that 
    $\mathbf{I}(\mathbf{V}(z-x^2-y^2)) = \langle z-x^2-y^2 \rangle$. 
    Let $f \in \mathbf{I}(\mathbf{V}(z-x^2-y^2))$ and divide it by 
    $z-x^2-y^2$ using lex order $z>x>y$. The result is that 
    $f=q(x,y,z)(z-x^2+y^2)+r(x,y)$. We can now substitute $x^2+y^2$ 
    for $z$ and have $r(x,y) = 0$. Since $\R$ is an infinite field, 
    $r(x,y)$ must be the zero polynomial, so $f \in \langle z-x^2-y^2 \rangle$.
    The other inclusion is obvious by observation, so we have that 
    $\mathbf{I}(\mathbf{V}(z-x^2-y^2))=\langle z-x^2-y^2\rangle$.

    Next, we let $V=\mathbf{V}(z-x^2-y^2)$. Then $\mathbf{V}_V([x-1],[y-1])$ 
    consists of the points in $V$ such that $x-1 \in \mathbf{I}(V)$ 
    and $y-1 \in \mathbf{I}(V)$. Note that $x-1,y-1$ are not divisible 
    by $z-x^2-y^2$, so it must be that $x=1,y=1$ and $z=1^2+1^2=2$. 
    Therefore $W=\{(1,1,2)\} = \mathbf{V}_V([x-1],[y-1])$. Using 
    Proposition 3.3, we have $\langle [x-1],[y-1] \rangle \subseteq \mathbf{I}_V(\mathbf{V}_V([x-1],[y-1]))=\mathbf{I}_V(W)$.
    
    \bigskip
    \textbf{(5.4.11b):}
    Define the mappings $\alpha:V\to \R^2$ and $\beta:\R^2\to V$ by 
    $\alpha(x,y,z)=(x,y)$ and $\beta(x,y)=(x,y,x^2+y^2)$, respectively.
    Then the compositions
    \begin{align*}
        (\alpha \circ \beta)(x,y) = \alpha(x,y,x^2+y^2)=(x,y),\\
        (\beta \circ \alpha)(x,y,z) = \beta(x,y)=(x,y,x^2+y^2)=(x,y,z), (x,y,z) \in V
    \end{align*}
    imply that $V$ is isomorphic as a variety to $\R^2$. Furthermore, 
    let $I=\mathbf{I}_V(W)\subseteq \R[V]$ and note that $W = \mathbf{V}_V(I)$
    by Proposition 5.5.3. Using Exercise 5.5.9 as per the given hint,
    we have $\alpha(W) = \mathbf{V}(\beta^*(I)),$ such that 
    $\beta^*(I) \subseteq \mathbf{I}(\alpha(W))$. Note that 
    $W=\{(1,1,2)\}$, so $\alpha(W) = \{(1,1)\}$, which yields 
    $\beta^*(I) \subseteq \mathbf{I}(\alpha(W))=\mathbf{I}(\{1,1\})=\langle x-1, y-1 \rangle$.

    Finally, since $\alpha^*:\R[x,y]\to\R[V]$ is a ring isomorphism 
    with an inverse $\beta^*$, there exists an injective, inclusion-preserving 
    correspondence between ideals of $\R[x,y]$ and $\R[V]$. Observe 
    that $\langle x-1, y-1 \rangle$ is the smallest ideal of 
    $\R[x,y]$ containing $x-1,y-1$, it follows that $\alpha^*(\langle x-1,y-1\rangle)$ 
    is the smallest ideal of $\R[V]$ which contains $\alpha^*(x-1)=[x-1]$ and 
    $\alpha^*(y-1)=[y-1].$ Thus $\alpha^*(\langle x-1,y-1\rangle)= \langle[x-1],[y-1]\rangle$,
    which implies 
    \[
        \mathbf{I}_V(W)=I=\alpha^*(\beta^*(I))\subseteq \alpha^*(\langle x-1, y-1\rangle)=\langle[x-1],[y-1]\rangle.    
    \]
    Therefore, we can conclude that $\langle[x-1],[y-1] \rangle=\mathbf{I}_V(W)$.

\end{exercise}

%%%%%%%%%%%%%%%%%%%%%%%%%%%%%%%%%%%%%%%%

%---------------------------------
% Don't change anything below here
%---------------------------------

\end{document}