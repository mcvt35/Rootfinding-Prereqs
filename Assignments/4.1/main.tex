\documentclass[12pt,oneside]{article}

% This package simply sets the margins to be 1 inch.
\usepackage[margin=1in]{geometry}

% These packages include nice commands from AMS-LaTeX
\usepackage{amssymb,amsmath,amsthm,graphicx}

% Define an environment for exercises.
\newenvironment{exercise}[1]{\vspace{.1in}\noindent\textbf{Exercise #1 \hspace{.05em}}}{}

% define shortcut commands for commonly used symbols
\newcommand{\R}{\mathbb{R}}
\newcommand{\C}{\mathbb{C}}
\newcommand{\Z}{\mathbb{Z}}
\newcommand{\Q}{\mathbb{Q}} 
\newcommand{\N}{\mathbb{N}} 

%%%%%%%%%%%%%%%%%%%%%%%%%%%%%%%%%%%%%%%%%%

\begin{document}

% If you use Overleaf, the name of the project will be determined by
% what you enter as the document title.
\title{Math Homework Template}

\begin{flushright}
\textsc{Marcelo Leszynski}  \\
Rootfinding Research Prerequisites\\
08/10/20
\end{flushright}

\begin{center}
\textsf{Assignment 4.1 } \\
\textsf{Exercises: 2, 4, 7, 8 }
\end{center}

%%%%%%%%%%%%%%%%%%%%%%%%%%%%%%%%%%%%%%%%

\begin{exercise}{4.1.2}
    We first begin by calculating $V(J)$ and note that, for any $(a,b) \in V(J)$,
    \[
        a^2+b^2-1 = b-1 = 0 \Rightarrow b = 1 \Rightarrow a = 0
    \]
    so $V(J) = \{(0,1)\}$. Using this variety, we see that the polynomial $f=x$ vanishes 
    at the point $(0,1)$, so $f \in I(V(J))$. To show that $f \not \in J$, we compute the 
    Groebner basis $G$ for $J$ with lex order and $x > y$ and have $G = \{x^2,y-1\}$. None 
    of the terms in $G$ divide $f = x$, so dividing $f$ by $G$ yields a nonzero remainder and
    $f = x \not \in \langle G \rangle = J$.
\end{exercise}

%%%%%%%%%%%%%%%%%%%%%%%%%%%%%%%%%%%%%%%%

\begin{exercise}{4.1.4}
    \begin{proof}
        Let $k$ be an algebraically closed field. We wish to show that $k$ must be infinite 
        and proceed by contradiction. Assume, by way of contradiction, that $k$ is finite 
        such that $k = \{a_1, \ldots, a_n\}$. We then have that the polynomial 
        $f = (x-a_1)\ldots(x-a_n) + 1$ is nonconstant and satisfies 
        $f(a_1) = \ldots = f(a_n) = 1$. Since $k$ is algebraically closed, we know that 
        there exists some $a \in k$ with $f(a) = 0$. We then have that $a = a_i$ for some 
        $1 \leq i \leq n$, which gives $0=f(a)=f(a_i)=1$, which is a contradiction. Therefore,
        $k$ must be infinite.
    \end{proof}
\end{exercise}

%%%%%%%%%%%%%%%%%%%%%%%%%%%%%%%%%%%%%%%%

\begin{exercise}{4.1.7}
    Credit to:
    
    \bigskip
    https://web.ma.utexas.edu/users/allcock/expos/nullstellensatz3.pdf
    
    \bigskip
    https://pdfs.semanticscholar.org/34b1/0b8c7deb313c12e729e1663a0b407dd3e80c.pdf
    
    \bigskip
    (2) $\Rightarrow$ (3): Let $k[t_1, \ldots, t_n]$ be a polynomial ring and let $L$ be 
    a field such that $k \subseteq L$. This implies that the terms $a_1, \ldots, a_n \in L$
    give a map 
    \[
        g(t_1,\ldots,t_n) \in k[t_1,\ldots,t_n] \rightarrow g(a_1,\ldots,a_n) \in L
    \]
    which preserves addition and multiplication. We apply this idea to the polynomial ring 
    $k[x_1,\ldots,x_n,y]$ over the field of rational functions $k(x_1,\ldots,x_n)$. We 
    can evaluate this ring at terms $x_1,\ldots,x_n,1/f$ with $1/f \in k(x_1,\ldots,x_n)$
    to yield the map 
    \[
        g(x_1,\ldots,x_n,y) \in k[x_1,\ldots,x_n,y] \rightarrow g(x_1,\ldots,x_n, 1/f) \in k(x_1,\ldots,x_n).
    \]
    Note that this mapping preserves addition and multiplication and is also the map of 
    the identity on $k[x_1,\ldots,x_n]$. Proceeding from equation (2) in the text, we can 
    evaluate the given polynomial as follows:
    \begin{align*}
        1   &= \sum_{i=1}^sp_1(x_1,\ldots,x_n,y)f_i+q(x_1,\ldots,x_n,y)(1-yf)\\
            &= \sum_{i=1}^sp_1(x_1,\ldots,x_n,1/f)f_i+q(x_1,\ldots,x_n,1/f)(1-(1/f)f)\\
            &= \sum_{i=1}^sp_1(x_1,\ldots,x_n,1/f)f_i+q(x_1,\ldots,x_n,1/f)\cdot 0\\
            &= \sum_{i=1}^sp_1(x_1,\ldots,x_n,1/f)f_i.
    \end{align*}
    The resulting equation given above can be used to justify equation (3) in the text.
    
    \bigskip
    (3) $\Rightarrow$ (4): To show this justification, we further analyze $p_i(x_1,\ldots,x_n,1/f)$.
    Note that 
    \[
        p_i(x_1,\ldots,x_n,y) = \sum_{l=1}^{m_i}h_l(x_1,\ldots,x_n)y^l.
    \]
    Using our previous mapping, this evaluates to 
    \begin{align*}
        p_i(x_1,\ldots,x_n,y)   &= p_i(x_1,\ldots,x_n,1/f) = \sum_{l=1}^{m_i}h_l(x_1,\ldots,x_n)1/f^l \in k(x_1,\ldots,x_n)\\
    \end{align*}
    It follows that 
    \[
        f^{m_i}p_i(x_1,\ldots,x_n,1/f) = \sum_{l=1}^{m_i}h_l(x_1,\ldots,x_n)f^{m_i-l}
    \]
    is a polynomial in $x_1,\ldots, x_n$. We can arrive at equation (4) by fixing 
    $m = \max(m_1,\ldots,m_s)$ and multiplying each side of (3) by $f^m$.
\end{exercise}

%%%%%%%%%%%%%%%%%%%%%%%%%%%%%%%%%%%%%%%%

\begin{exercise}{4.1.8}

    \bigskip
    \textbf{(4.1.8a):}
    \begin{proof}
        Let $g = a_0x^n + a_1x^{n-1} + \ldots + a_{n-1}x + a_n$ be a polynomial of degree 
        $n$ in $x$, and define the homogenization $g^h$ of $g$ with respect to some 
        variable $y$ as the polynomial $g^h = a_0x^n +a_1x^{n-1}y + \ldots a_{n-1}xy^{n-1} + a_ny^n$.
        We wish to show that $g$ has a root in $k$ if and only if there exists some 
        $(a,b) \in k^2$ such that $(a,b) \neq (0,0)$ and $g^h(a,b) = 0$ Before we 
        proceed, we note that 
        \[
            g^h(x,y) = \sum_{i=0}^n a_ix^{n-i}y^i = \sum_{i=0}^na_i\biggr(\frac{x}{y}\biggr)^{n-i}y^n = y^ng^h\biggr(\frac{x}{y},1\biggr) = y^ng\biggr(\frac{x}{y}\biggr).
        \]
        
        \bigskip
        ($\Rightarrow$): Assume that $g$ has a root $r \in k$. Using the above notation, 
        we have that 
        \[
            0 = g(r) = 1^ng(r/1) = g^h(r,1),
        \]
        so $(r,1)$ is a root of $g^h$.
        
        \bigskip
        ($\Leftarrow$): Assume that there is some nontrivial root $(r_1,r_2) \neq (0,0)$ 
        of $g^h$. We now divide the proof into two cases: in the first case, $r_2 = 0$. 
        Then $0=g^h(r_1,0) = a_0r_1^n$, implying that $r_1 = 0$. This is a contradiction 
        since $(r_1,r_2) \neq (0,0)$, so this case can never hold. 
        In the second case, $r_2 \neq 0$, so we have 
        \[
            0 = g^h(r_1,r_2) = r_2^ng(r_1/r_2),
        \]
        implying that $g(r_1,r_2) = 0$.
    \end{proof}
    
    \bigskip
    \textbf{(4.1.8b):}
    \begin{proof}
        Let $g$ be a nonconstant polynomial with no root in $k$ of degree $n$. Let 
        $f(x,y) = g^h$ be the homogenization of $g$. Then $f$ vanishes at $(0,0)$ since 
        every term of $f$ is of degree $n > 0$. We know by Exercise 4.1.8a that any other 
        root $(a,b)$ of $f$ with $f(a,b) = 0, (a,b) \neq 0$ implies that $g$ has a root 
        in $k$. By construction of $g$, we know that this cannot occur. Therefore $(0,0)$ 
        is the only solution of $f=0$.
    \end{proof}
    
    \bigskip
    \textbf{(4.1.8c):}
    \begin{proof}
        
        \bigskip
        \underline{Base Case}:
        Let $f_2 = f$ be given by $f$ in Exercise 4.1.8b. This proves the base case for 
        our inductive proof of $n=2$. 
        
        \bigskip
        \underline{Inductive Hypothesis}:
        We proceed by induction and assume there exists some
        $f_m \in k[x_1,\ldots,x_m]$ such that $f_m(a_1,\ldots, a_m) = 0$ if and only if 
        $a_1 = \ldots = a_m = 0$ for $(a_1,\ldots, a_m) \in k^m$. 
        
        \bigskip
        \underline{Inductive Step}:
        Define $f_{m+1} = f_2(f_m(x_1,\ldots,x_m),x_{m+1})$ and let 
        $(a_1,\ldots,a_m,b) \in k^{m+1}$. We then have that 
        \begin{align*}
            0 = f_{m+1}(a_1,\ldots,a_m,b) \Leftrightarrow 0 = f_2(f_m(a_1,\ldots,a_m)b) &\Leftrightarrow f_m(a_1,\ldots,a_m) = 0 \text{ and } b=0\\
            &\Leftrightarrow a_1 = \ldots = a_m = 0 \text{ and } b=0.\\
        \end{align*}
    \end{proof}
    
    \newpage
    \textbf{(4.1.8d):}
    \begin{proof}
        Let $W = V(g_1,\ldots,g_s)$ be a variety in $k^n$ with $k$ not being algebraically 
        closed, and let $f_s \in k[y_1,\ldots,y_s]$ be given by Exercise 4.1.8c. Note 
        the existence of $f_s$ is due to $k$ not being algebraically closed. We can set 
        \begin{align*}
            h&=f_s(g_1,\ldots,g_s) \in k[x_1,\ldots,x_n],\\
            a&=(a_1,\ldots,a_n) \in k^n
        \end{align*}
        which gives 
        \[
            h(a) = 0 \Leftrightarrow f_s(g_1(a),\ldots,g_s(a)) = 0 \Leftrightarrow g_1(a) = \ldots = g_s(a) = 0.
        \]
        This proves that $W = V(h)$ as desired.
    \end{proof}
\end{exercise}

%%%%%%%%%%%%%%%%%%%%%%%%%%%%%%%%%%%%%%%%

%---------------------------------
% Don't change anything below here
%---------------------------------

\end{document}