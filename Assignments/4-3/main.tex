\documentclass[12pt,oneside]{article}

% This package simply sets the margins to be 1 inch.
\usepackage[margin=1in]{geometry}

% These packages include nice commands from AMS-LaTeX
\usepackage{amssymb,amsmath,amsthm,graphicx}

% Define an environment for exercises.
\newenvironment{exercise}[1]{\vspace{.1in}\noindent\textbf{Exercise #1 \hspace{.05em}}}{}

% define shortcut commands for commonly used symbols
\newcommand{\R}{\mathbb{R}}
\newcommand{\C}{\mathbb{C}}
\newcommand{\Z}{\mathbb{Z}}
\newcommand{\Q}{\mathbb{Q}} 
\newcommand{\N}{\mathbb{N}} 

%%%%%%%%%%%%%%%%%%%%%%%%%%%%%%%%%%%%%%%%%%

\begin{document}

% If you use Overleaf, the name of the project will be determined by
% what you enter as the document title.
\title{Math Homework Template}

\begin{flushright}
\textsc{Marcelo Leszynski}  \\
Rootfinding Research Prerequisites\\
08/10/20
\end{flushright}

\begin{center}
\textsf{Assignment 4.3 } \\
\textsf{Exercises: 2, 3, 4, 6, 8 }
\end{center}

%%%%%%%%%%%%%%%%%%%%%%%%%%%%%%%%%%%%%%%%

\begin{exercise}{4.3.2}
    \begin{proof}
        Let $f,g \in k[x_1,\ldots,x_2]$ and let $f=cf_1^{a_1}\ldots f_r^{a_r}$ and 
        $g=c^lg_1^{b_1}\ldots g_s^b$ be their factorizations into distinct irreducible 
        polynomials such that $f_1,\ldots,f_l$ are constant multiples of 
        $g_1,\ldots,g_l$ and for $i,j>l, f_i$ is not a constant multiple of $g_j$. 
        We wish to show that 
        \[
            \text{lcm}(f,g) = f_1^{\max(a_1,b_1)}\ldots f_l^{\max(a_l,b_l)}\cdot g_{l+1}^{b_{l+1}}\ldots g_s^{b_s}\cdot f_{l+1}^{a_{l+1}}\ldots f_r^{a_r}.
        \]
        
        We begin by noting that lcm$(f,g)$ is a common multiple of $f,g$ by definition. 
        We can define another arbitrary common multiple $h$ of $f$ and $g$, which implies 
        that the irreducible factorization of $h$ includes all irreducible factors of $f,g$.
        In other words, $h$ contains all 
        $f_1 = g_1,\ldots,f_l = g_l, f_{l+1},\ldots,f_r,g_{l+1},\ldots,g_s$. Since the 
        exponent of each of these irreducible factors of $h$ must be greater than or equal 
        to the corresponding exponents in $f$ and $g$, it follows that lcm$(f,g)$ must 
        divide $h$. Therefore, we can conclude that lcm$(f,g)$ as defined above is 
        the least common multiple of $f$ and $g$.
    \end{proof}
\end{exercise}

%%%%%%%%%%%%%%%%%%%%%%%%%%%%%%%%%%%%%%%%

\begin{exercise}{4.3.3}
    \begin{proof}
        Let $I,J \subseteq k[x_1,\ldots,x_n]$ be principal ideals. We wish to show that 
        their intersection $I \cap J$ is also a principal ideal. To show this, we let 
        $h \in I \cap J, I= \langle f \rangle, J= \langle g \rangle$ as defined in 
        Proposition 13. We then have that 
        \begin{align*}
            h \in I \cap J  &\Leftrightarrow h\in I, h\in J\\
                            &\Leftrightarrow h \in \langle f \rangle, h \in \langle g \rangle\\
                            &\Leftrightarrow h \text{ is a multiple of }f, g\\
                            &\Leftrightarrow h \text{ is a multiple of lcm}(f,g) && \text{by Exercise 4.3.2}\\
                            &\Leftrightarrow h \in \langle \text{lcm}(f,g)\rangle.
        \end{align*}
        From this equivalence, it follows that 
        $\langle f \rangle \cap \langle g \rangle = \langle$lcm$(f,g)\rangle$. Therefore, 
        the intersection of two principal ideals is also principal.
    \end{proof}
\end{exercise}

%%%%%%%%%%%%%%%%%%%%%%%%%%%%%%%%%%%%%%%%

\begin{exercise}{4.3.4}
    \begin{proof}
        Let $I,J \subseteq k[x_1,\ldots,x_n]$ be principal ideals and assume that 
        $I=\langle f \rangle, J = \langle g \rangle$, and $I\cap J = \langle h \rangle$. 
        We wish to show that $h = $lcm$(f,g)$. Note that this equivalence is demonstrated 
        by the equivalences given in Exercise 4.3.3, i.e., 
        $I\cap J = \langle\text{lcm}(f,g)\rangle$, together with our assumption, implies that 
        $h = $ lcm$(f,g)$.
    \end{proof}
\end{exercise}

%%%%%%%%%%%%%%%%%%%%%%%%%%%%%%%%%%%%%%%%
\newpage
\begin{exercise}{4.3.6}
    
    \bigskip
    \textbf{(4.3.6a):}
    Let $I_l,\ldots, I_r, J$ be ideals in $k[x_1,\ldots,x_n]$. We wish to show that 
    $(I_1+I_2)J = I_1J+I_2J$. Defining these ideals in terms of their generators, we have 
    $I_1 = \langle f_1,\ldots, f_r \rangle, I_2 = \langle g_1,\ldots, g_s \rangle$, and 
    $J = \langle h_1, \ldots, h_t \rangle$. The definition of ideal multiplication then 
    implies that 
    \begin{align*}
        (I_1+I_2)J  &= \langle f_1,\ldots,f_r,g_1,\ldots,g_s\rangle \cdot \langle h_1,\ldots,h_t \rangle\\
                    &= \langle f_ih_l,g_jh_l : 1\leq i \leq r, 1 \leq j \leq s, 1 \leq l \leq t \rangle\\
                    &\text{and}\\
        I_1J+I_2J   &= \langle f_ih_l : 1 \leq i \leq r, 1 \leq l \leq t \rangle + \langle g_jh_l : 1 \leq j \leq s, 1 \leq l \leq t \rangle\\
                    &= \langle f_1h_l,g_jh_l: 1 \leq i \leq r, 1 \leq j \leq s, 1 \leq l \leq t \rangle.
    \end{align*}
    Therefore, $(I_1+I_2)J = I_1J + I_2J$ as needed.
    
    \bigskip
    \textbf{(4.3.6b):}
    Let $I_l,\ldots, I_r, J$ be ideals in $k[x_1,\ldots,x_n]$. We wish to show that 
    $(I_1\ldots I_r)^m = I_1^m\ldots I_r^m$ and proceed by induction on $m \geq 1$. 
    
    \underline{Base Case}: This case is easily confirmed by observation
    
    \underline{Inductive Hypothesis}: Assume that $(I_1\ldots I_r)^k = I_1^k\ldots I_r^k$ 
    for all $k\geq 1$. 
    
    \underline{Inductive Step}: The inductive hypothesis implies that 
    \[
        (I_1\ldots I_r)^{k+1} = (I_1\ldots I_r)^k \cdot I_1\ldots I_r = I_1^k\ldots I_r^k\cdot I_1 \ldots I_r,
    \]
    which implies that 
    \[
        (I_1\ldots I_r)^{k+1} = (I_1^k \cdot I_1) \ldots (I_r^k \cdot I_r) = I_1^{k+1} \ldots I_r^{k+1}
    \]
    as needed.
\end{exercise}

%%%%%%%%%%%%%%%%%%%%%%%%%%%%%%%%%%%%%%%%

\begin{exercise}{4.3.8}
    
    \bigskip
    \textbf{(4.3.8a):}
    To compute generators for $\langle f \rangle \cap \langle g \rangle$ with $f,g$ being 
    defined in the text, we use Theorem 4.3.11 and compute a lex order Groebner basis for
    $\langle tf, (1-t)g \rangle$ with $t>x>y>z$. This turns out to be 
    
    \begin{align*}
    \{&x^5 + x^4 y - x^3 y^2 - x^2 y^3 + 2 x^4 z^2 + 2 x^3 y z^2 2 x^2 y^2 z^2 - 2 x y^3 z^2 + x^3 z^4 + x^2 y z^4 - x y^2 z^4 y^3 z^4,\\&x^4 + t x^3 y - x^2 y^2 - t x y^3 + 2 x^3 z^2 - t x^3 z^2 + t x^2 y z^2 \\&- 2 x y^2 z^2 + t x y^2 z^2 - t y^3 z^2 + x^2 z^4 t x^2 z^4 - y^2 z^4 + t y^2 z^4,\\&-x^4 + t x^4 + x^2 y^2 t x^2 y^2 - 2 x^3 z^2 + 2 t x^3 z^2 + 2 x y^2 z^2 - 2 t x y^2 z^2 x^2 z^4 + t x^2 z^4 + y^2 z^4 - t y^2 z^4\}.
    \end{align*}
    
    Note that all polynomials in this set involve $t$ except for the first. This means that 
    a basis for $\langle f \rangle \cap \langle g \rangle$ can be given by 
    \[
        x^5+x^4y-x^3y^2-x^2y^3+2x^4z^2+2x^3yz^2-2x^2y^2z^2-2xy^3z^2+x^3z^4+x^2yz^4-xy^2z^4-y^3z^4.
    \]
    Note that this polynomial is the same as lcm$(f,g)$ by Proposition 4.3.13.
    
    To compute generators for $\sqrt{\langle f \rangle \langle g \rangle}$, note that 
    $\sqrt{\langle f \rangle \langle g \rangle} = \sqrt{\langle fg \rangle}$. We can 
    use Proposition 4.2.12 to equate 
    $\sqrt{\langle fg \rangle} = \langle (fg)_{red} \rangle$, where 
    \[
        (fg)_{red} = \frac{fg}{\text{gcd}(fg,\frac{\partial fg}{\partial x},\frac{\partial fg}{\partial y},\frac{\partial fg}{\partial z})}.
    \]
    
    Since we know that
    \[
        \text{gcd}\biggr(fg,\frac{\partial fg}{\partial x},\frac{\partial fg}{\partial y},\frac{\partial fg}{\partial z}\biggr) = \frac{fg\cdot\frac{\partial fg}{\partial x}\cdot\frac{\partial fg}{\partial y}\cdot\frac{\partial fg}{\partial z}}{\text{lcm}(fg,\frac{\partial fg}{\partial x},\frac{\partial fg}{\partial y},\frac{\partial fg}{\partial z})}
    \]
    by Proposition 4.3.14, we have that 
    $\sqrt{\langle f \rangle \langle g \rangle} = \langle -x^3+xy^2-x^2z^2+y^2z^2 \rangle$.
        
    \bigskip
    \textbf{(4.3.8b):}
        Using Proposition 4.3.14, we have that 
        \[
            \text{gcd}(f,g) = \frac{fg}{\text{lcm}(f,g)} = x^3-xy^2+x^2z^2-y^2z^2.
        \]
        
    \bigskip
    \textbf{(4.3.8c):}
    Using the definitions for $p,q$ found in the text, we compute a lex order Groebner basis
    of $\langle tf,tg,(1-t)p,(1-t)q \rangle$ with $t,x,y,z$ and get
    
    \begin{align*}
    \{&x^3 y - x y^3 - x^3 z^2 + x^2 y z^2 + x y^2 z^2 y^3 z^2 - x^2 z^4 + y^2 z^4,\\&x^4 - x^2 y^2 + 2 x^3 z^2 - 2 x y^2 z^2 + x^2 z^4 - y^2 z^4,\\&-y^2 z + t y^2 z - y z^2 + t y z^2,\\&-x y + t x y - x z + t x z,\\&-x^2 + t x^2 - y z + t y z\}. 
    \end{align*}
    
    Since the first two polynomials in this basis don't involve $t$, we have that 
    \begin{align*}
        \langle f,g \rangle \cap \langle p,q \rangle = &\langle x^3y-xy^3-x^3z^2+x^2yz^2+xy^2z^2-y^3z^2-x^2z^4+y^2z^4,\\&x^4-x^2y^2+2x^3z^2-2xy^2z^2+x^2z^4-y^2z^4 \rangle.
    \end{align*}
\end{exercise}

%%%%%%%%%%%%%%%%%%%%%%%%%%%%%%%%%%%%%%%%

%---------------------------------
% Don't change anything below here
%---------------------------------

\end{document}