\documentclass[12pt,oneside]{article}

% This package simply sets the margins to be 1 inch.
\usepackage[margin=1in]{geometry}

% These packages include nice commands from AMS-LaTeX
\usepackage{amssymb,amsmath,amsthm,graphicx}

% Define an environment for exercises.
\newenvironment{exercise}[1]{\vspace{.1in}\noindent\textbf{Exercise #1 \hspace{.05em}}}{}

% define shortcut commands for commonly used symbols
\newcommand{\R}{\mathbb{R}}
\newcommand{\C}{\mathbb{C}}
\newcommand{\Z}{\mathbb{Z}}
\newcommand{\Q}{\mathbb{Q}}
\newcommand{\N}{\mathbb{N}}

%%%%%%%%%%%%%%%%%%%%%%%%%%%%%%%%%%%%%%%%%%

\begin{document}

% If you use Overleaf, the name of the project will be determined by
% what you enter as the document title.
\title{Math Homework Template}

\begin{flushright}
\textsc{Marcelo Leszynski}  \\
Rootfinding Research Prerequisites\\
07/30/20
\end{flushright}

\begin{center}
\textsf{Assignment 2.3 } \\
\textsf{Exercises: 1, 3, 5 }
\end{center}

%%%%%%%%%%%%%%%%%%%%%%%%%%%%%%%%%%%%%%%%

\begin{exercise}{2.3.1}

    \bigskip
    \textbf{(2.3.1a):}
    Using grlex order yields
    \[
        f=(x^6+x^2)\cdot(xy^2-x)+0\cdot(x-y^3)+(x^7+x^3-y+1)   
    \]
    whereas using lex order yields
    \begin{align*}
        f=&(x^6+x^5y+x^4y^2+x^4+x^3y+x^2y^2+2x^2+2xy+2y^2+2)\cdot(xy^2-x)+\\
        &(x^6+x^5y+x^4+x^3y+2x^2+2xy+2)\cdot(x-y^3)+(2y^3-y+1)   
    \end{align*}

    \bigskip
    \textbf{(2.3.1b):}
    Using grlex order yields
    \[
        f=0\cdot(x-y^3)+(x^6+x^2)\cdot(xy^2-x)+(x^7+x^3-y+1)    
    \]
    whereas using lex order yields
    \begin{align*}
        f=&(x^6y^2+x^5y^5+x^4y^8+x^3y^11+x^2y^14+x^2y^2+xy^17+xy^5+y^20+y^8)\cdot(x-y^3)+\\
        &0\cdot(xy^2-x)+(y^23+y^11-y+1)
    \end{align*}

\end{exercise}

%%%%%%%%%%%%%%%%%%%%%%%%%%%%%%%%%%%%%%%%

\begin{exercise}{2.3.3}
    Implementation of the division algorithm can be found in poly\textunderscore div.py
    in the Coded Algorithms folder.
\end{exercise}

%%%%%%%%%%%%%%%%%%%%%%%%%%%%%%%%%%%%%%%%

\begin{exercise}{2.3.5}

    \bigskip
    \textbf{(2.3.5a):}
    using poly\textunderscore div.py to calculate $r_1$ and $r_2$ yields 
    \begin{align*}
        r_1 &= x^3-x^2z+x-z\\
        r_2 &= x^3-x^2z
    \end{align*}
    The difference between the two calculations occurs during the second 
    step of the algorithm when we are comparing leading terms.

    \bigskip
    \textbf{(2.3.5b):}
    \[
        r = r_1-r_2 = (x^3-x^2z+x-z)-(x^3-x^2z)=x-z \in \langle f_1, f_2 \rangle   
    \]
    because $x-z = f_1-xf_2$.

    \bigskip
    \textbf{(2.3.5c):}
    The remainder is equal to $x-z$. This could be predicted because the 
    leading terms of $f_1$ and $f_2$ do not divide any of the terms in $r$.

    \bigskip
    \textbf{(2.3.5d):}
    Let $g=-y\cdot f_1+(xy+1)\cdot f_2 = yz-1 \in \langle f_1, f_2 \rangle$. 
    note that no term in $g$ is divided by the leading terms of $f_1$ and $f_2$.

    \bigskip
    \textbf{(2.3.5e):}
    The division algorithm does not solve the ideal membership problem since 
    both $r$ and $g$ are contained in $\langle f_1, f_2 \rangle$ but dividing 
    using the algorithm returns nonzero remainders for both $r$ and $g$.

\end{exercise}

%%%%%%%%%%%%%%%%%%%%%%%%%%%%%%%%%%%%%%%%

%---------------------------------
% Don't change anything below here
%---------------------------------

\end{document}