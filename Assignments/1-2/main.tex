\documentclass[12pt,oneside]{article}

% This package simply sets the margins to be 1 inch.
\usepackage[margin=1in]{geometry}

% These packages include nice commands from AMS-LaTeX
\usepackage{amssymb,amsmath,amsthm,graphicx}


% Define an environment for exercises.
\newenvironment{exercise}[1]{\vspace{.1in}\noindent\textbf{Exercise #1 \hspace{.05em}}}{}

% define shortcut commands for commonly used symbols
\newcommand{\R}{\mathbb{R}}
\newcommand{\C}{\mathbb{C}}
\newcommand{\Z}{\mathbb{Z}}
\newcommand{\Q}{\mathbb{Q}}
\newcommand{\N}{\mathbb{N}}


%%%%%%%%%%%%%%%%%%%%%%%%%%%%%%%%%%%%%%%%%%

\begin{document}

% If you use Overleaf, the name of the project will be determined by
% what you enter as the document title.
\title{Math Homework Template}

\begin{flushright}
\textsc{Marcelo Leszynski}  \\
Rootfinding Research Prerequisites\\
05/17/20
\end{flushright}

\begin{center}
\textsf{Assignment 1.2 } \\
\textsf{Exercises: 1, 2, 3, 4, 5, 6, 8, 9, 15 }
\end{center}

%%%%%%%%%%%%%%%%%%%%%%%%%%%%%%%%%%%%%%%%

\begin{exercise}{1.2.1}

\bigskip
\textbf{(1.2.1a)}

\includegraphics[width=5cm]{1a.png}

\bigskip
\textbf{(1.2.1b)}

\includegraphics[width=5cm]{1b.png}

\bigskip
\textbf{(1.2.1c)}

\includegraphics[width=5cm]{1c.png}

\end{exercise}

%%%%%%%%%%%%%%%%%%%%%%%%%%%%%%%%%%%%%%%%
\newpage
\begin{exercise}{1.2.2}

\includegraphics[width=5cm]{2.png}
\end{exercise}

%%%%%%%%%%%%%%%%%%%%%%%%%%%%%%%%%%%%%%%%

\begin{exercise}{1.2.3}
    The following is $V(x^2+y^2-4)\cap V(xy-2)$, which is the 
    intersection of the two affine varieties represented by 
    two different colors. The points of intersection are 
    approximately $(0.518,1.932),(1.932,0.518),(-0.518,-1.932),
    (-1.932,-0.518)$.

    \includegraphics[width=5cm]{3a.png}

    These points of intersection are equal to the affine variety 
    $V(x^2+y^2-4,xy-1)$.

    \includegraphics[width=5cm]{3b.png}
\end{exercise}

%%%%%%%%%%%%%%%%%%%%%%%%%%%%%%%%%%%%%%%%
\newpage
\begin{exercise}{1.2.4}

    \bigskip
    \textbf{(1.2.4a)}
    A unit sphere centered at the origin.

    \bigskip
    \textbf{(1.2.4b)}
    A cylinder whose cross-sections parallel to the $xy$-plane are unit circles centered at 
    (0,0) in the $xy$-plane.

    \bigskip
    \textbf{(1.2.4c)}
    A point $(-2,1.5,0)$.

    \bigskip
    \textbf{(1.2.4d)}
    The union of the $yz$-plane and a parabolic cylinder perpendicular to the $yz$-plane above 
    $y=z^2$.

    \bigskip
    \textbf{(1.2.4e)}
    The union of the $yz$-plane and a twisted cubic given by $V(x^3-z,x^2-y)$.

    \bigskip
    \textbf{(1.2.4f)}
    A circle with radius $\frac{\sqrt{3}}{2}$ centered at $(0,0,\frac{1}{2})$ that lies on 
    the plane $z = \frac{1}{2}$.
\end{exercise}

%%%%%%%%%%%%%%%%%%%%%%%%%%%%%%%%%%%%%%%%

\begin{exercise}{1.2.5}
    The affine variety in $\R^3$ given by $V((x-2)(x^2-y),y(x^2-y),(z+1)(x^2-y))$ can be
    described as the union $V(x-2,y,z+1)\cup V(x^2-y)$, which is the union of the point 
    $(2,0,-1)$ and a parabolic cylinder which is perpendicular to the $xy$-plane above the 
    parabola $y=x^2$.
\end{exercise}

%%%%%%%%%%%%%%%%%%%%%%%%%%%%%%%%%%%%%%%%

\begin{exercise}{1.2.6}

    \bigskip
    \textbf{(1.2.6a)}
    \begin{proof}
        Define $V$ to be the affine variety $V(f_1,\ldots,f_n)$ with each $f_i = x_i - a_i$ for
        all $1 \leq i \leq n$. Then $V = \{(b_1, \ldots, b_n) | f_i(b_1,\ldots, b_n) = 0, 1\leq i \leq n\}$. Thus we have that $(b_1,\ldots, b_n) \in V$ implies $b_i - a_i = 0$ for $1 \leq i \leq n$. This results in $(b_1, \ldots, b_n) = (a_1, \ldots, a_n)$ so $V = \{(a_1, \ldots, a_n)\}$. 

        Since this works for an arbitrary point $(a_1, \ldots, a_n)$, this argument will 
        hold for any point. Therefore, a single point $(a_1, \ldots, a_n) \in k^n$ is an affine
        variety.
    \end{proof}

    \bigskip
    \textbf{(1.2.6b)}
    \begin{proof}
        Let $S$ be a finite subset of $k^n$ with $r$ points such that $S = \{p_1, \ldots, p_r\}$, $p_i \in k^n$ for $1 \leq i \leq r$. 

        By (1.2.6a), we have that each set of one point $\{p_i\}$ is an affine variety $V_i$. 
        Lemma 2 indicates that finite unions of affine varieties are also affine varieties, so
        taking the union of finitely many $p_i$ yields
        \[
            V = \bigcup_{i=1}^r V_i = \bigcup_{i=1}^r \{p_i\} = S.
        \]

        Since $S$ is arbitrary, we conclude that every finite subset $S$ of a field $k^n$ is an
        affine variety.
    \end{proof}
\end{exercise}

%%%%%%%%%%%%%%%%%%%%%%%%%%%%%%%%%%%%%%%%
\newpage
\begin{exercise}{1.2.8}
    \begin{proof}
        Let $f(x,y) \in \R[x,y]$ vanish on $X = \{(x,x) \in \R^2 | x \neq 1\}$. Then 
        $g(t) = f(t,t) \in \R[t]$ vanishes at all points such that $t = x \neq 1$. This 
        implies that $g(t)$ has infinitely many roots, so $g(t)$ must therefore be the 
        zero polynomial. Note, however, that if $g(t) = 0$ is the zero polynomial, then 
        $g(1) = f(1,1) = 0$. Thus, we conclude that $X$ is not an affine variety.
    \end{proof}
\end{exercise}

%%%%%%%%%%%%%%%%%%%%%%%%%%%%%%%%%%%%%%%%

\begin{exercise}{1.2.9}
    \begin{proof}
        Assume, by way of contradiction, that $R = \{(x,y) \in \R^2 | y > 0\}$ is a 
        variety. Then $R = V(f_1,\ldots f_n)$ where each $f_i$ is a polynomial in $\R[x,y]$.
        We consider $f_1(x,y)$ and let $y = y_0 > 0$. Note that 
        \[
            f_1(x, y_0) = \sum_{i=0}^N g_i(y_0)x^i
        \]
        vanishes at all $x \in \R$. This means that $g_i(y_0) = 0$ is the zero polynomial 
        for all $i$. This holds for all $y_0 > 0$. Thus, $f_1$ is the zero polynomial. This 
        can be applied to all $f_i$ so $f_i = 0$ is the zero polynomial for all $i$. 
        Now $R = V(0, \ldots 0) = \R^2$, and we arrive at a contradiction. Therefore
        $R$ is not an affine variety.
    \end{proof}
\end{exercise}

%%%%%%%%%%%%%%%%%%%%%%%%%%%%%%%%%%%%%%%%

\begin{exercise}{1.2.15}

    \bigskip
    \textbf{(1.2.15a)}
    \begin{proof}
        The base cases have been shown for both unions and intersections of two affine varieties. Thus, we can rewrite finite intersections and unions as follows:
        $V_1 \cup \ldots \cup V_m = (V_1 \cup \ldots \cup V_{m-1}) \cup V_m$ and 
        $V_1 \cap \ldots \cap V_m = (V_1 \cap \ldots \cap V_{m-1}) \cap V_m$, so 
        our result follows from induction on $m$.
    \end{proof}
    
    \bigskip
    \textbf{(1.2.15b)}
    \begin{proof}
        Consider the affine variety $V_i = \{i\} \subseteq \R$. Note that 
        $\bigcup_{i = 1}^\infty = \N$. Note that this set is infinite, but 
        not equal to $\R$, so it is not an affine variety.
    \end{proof}
    
    \bigskip
    \textbf{(1.2.15c)}
    \begin{proof}
        Define $V = \R, W = \{0\}$. Then $V\backslash W = \R - \{0\}$ which is an infinite 
        set not equal to $\R$. Thus, similarly to (1.2.15b), $V\backslash W$ is not an affine 
        variety.
    \end{proof}
    
    \newpage
    \textbf{(1.2.15d)}
    \begin{proof}
        Let $V = V(f_1,\ldots, f_s)$ with $f_i \in k[x_1, \ldots, x_n]$ and let 
        $W = V(g_1, \ldots, g_t)$ with $g_j \in k[x_1, \ldots, x_m]$. Furthermore, 
        define $y_1, \ldots, y_m$ to be new variables and $h_j = g_j(y_1,\ldots, y_m)$.
        
        We wish to show that 
        \[
            V \times W = V(f_1, \ldots, f_s, h_1, \ldots, h_t) \subseteq k^n \times k^m
        \]
        
        Note that, if $f(a,b) \in V \times W$, then $f_i(a,b) = f_i(a) = 0$ and 
        $h_j(a,b) = h_j(b) = 0$ since $a \in V, b \in W$. Thus $(a,b) \in V(f_1, \ldots, f_s, h_1, \ldots, h_t)$.
        
        Conversely, if $(a,b) \in V(f_1, \ldots, f_s, h_1, \ldots, h_t)$, then $f_i(a) = 0$ 
        for $i = 1, \ldots, s$, and $g_j(b) = 0$ for $j = 1, \ldots, t$. Therefore $a \in V$
        and $b \in W$.
        
        Thus we conclude that the cartesian product of two affine varieties is an affine 
        variety.
    \end{proof}
\end{exercise}

%%%%%%%%%%%%%%%%%%%%%%%%%%%%%%%%%%%%%%%%

%---------------------------------
% Don't change anything below here
%---------------------------------


\end{document}
