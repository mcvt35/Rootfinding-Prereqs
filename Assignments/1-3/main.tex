\documentclass[12pt,oneside]{article}

% This package simply sets the margins to be 1 inch.
\usepackage[margin=1in]{geometry}

% These packages include nice commands from AMS-LaTeX
\usepackage{amssymb,amsmath,amsthm,graphicx}


% Define an environment for exercises.
\newenvironment{exercise}[1]{\vspace{.1in}\noindent\textbf{Exercise #1 \hspace{.05em}}}{}

% define shortcut commands for commonly used symbols
\newcommand{\R}{\mathbb{R}}
\newcommand{\C}{\mathbb{C}}
\newcommand{\Z}{\mathbb{Z}}
\newcommand{\Q}{\mathbb{Q}}
\newcommand{\N}{\mathbb{N}}


%%%%%%%%%%%%%%%%%%%%%%%%%%%%%%%%%%%%%%%%%%

\begin{document}

% If you use Overleaf, the name of the project will be determined by
% what you enter as the document title.
\title{Math Homework Template}

\begin{flushright}
\textsc{Marcelo Leszynski}  \\
Rootfinding Research Prerequisites\\
05/17/20
\end{flushright}

\begin{center}
\textsf{Assignment 1.3 } \\
\textsf{Exercises: 1, 2, 3, 4, 6, 7, 11, 13, 14 }
\end{center}

%%%%%%%%%%%%%%%%%%%%%%%%%%%%%%%%%%%%%%%%

\begin{exercise}{1.3.1}
    Using row reduction, we can arrive at the system of equations
    \[
        \begin{cases}
            x + 4z -3w = 5\\
            y - 3x + 2w = -3.
        \end{cases}
    \]
    From here, we let $z = t, w = u$ where $t$ and $u$ are arbitrary parameters, and 
    get the parametrization 
    \[
        \begin{cases}
            x = 5 - 4t + 3u\\
            y = -3 + 3t - 2u\\
            z = t\\
            w = u.
        \end{cases}
    \]
\end{exercise}

%%%%%%%%%%%%%%%%%%%%%%%%%%%%%%%%%%%%%%%%

\begin{exercise}{1.3.2}
    Using the trigonometric identity $\cos(2t) = 2\cos^2(t) - 1$, we have that 
    $x = cos(t)$, $y = \cos(2t) = 2\cos^2(t) - 1$. Substituting $x$ for $\cos(t)$ into $y$
    yields the parabola $y=2x^2-1$. This parametrization covers the parabola contained 
    in the square $-1 \leq x, y \leq 1$ since $-1 \leq \cos(\theta) \leq 1$.
\end{exercise}

%%%%%%%%%%%%%%%%%%%%%%%%%%%%%%%%%%%%%%%%

\begin{exercise}{1.3.3}
    This parametrization can be given by $x = t, y = f(t)$.
\end{exercise}

%%%%%%%%%%%%%%%%%%%%%%%%%%%%%%%%%%%%%%%%

\begin{exercise}{1.3.4}

    \bigskip
    \textbf{(1.3.4a)}
    We begin by solving $x = \frac{t}{1+t}$ for $t$, which yields $t = \frac{x}{1-x}$. Next,
    we substitute this value into $y = 1 - \frac{1}{t^2}$ so
    \[
        y = 1 - \frac{1}{t^2} = 1 - \frac{(1-x)^2}{x^2} = \frac{2x-1}{x^2}.
    \]
    Thus we have that $x^2y = 2x-1$, which implies that the parametrization is contained in
    $V(x^2y-2x+1)$.
    
    \bigskip
    \textbf{(1.3.4b)}
    First, note that $(1,1) \in V(x^2y-2x+1)$ and that $1 \neq \frac{t}{1+t}$ for all 
    $t \in \R$. Thus, the parametrization excludes $(1,1)$. To see all other points are 
    covered by the parametrization, suppose $(x,y)$ satisfies $x^2y - 2x + 1 = 0$ with 
    $x \neq 1$. Observe now that $x \neq 0$, since $0 * y - 2 * 0 + 1 = 0$ is impossible. 
    We therefore set $t = \frac{x}{1-x} \in \R$ and observe that $t \neq 0$ due to our 
    constraints on $x$. Then $x = \frac{t}{1+t}$ by our definition of $t$ and 
    \[
        1 - \frac{1}{t^2} = 1 - \frac{(1 - x)^2}{x^2} = \frac{2x - 1}{x^2} = y.
    \]
    Note that the last equality follows from $x^2y - 2x + 1 = 0$. Therefore, $(x,y)$ comes 
    from the parametrization.
\end{exercise}

%%%%%%%%%%%%%%%%%%%%%%%%%%%%%%%%%%%%%%%%

\begin{exercise}{1.3.6}

    \bigskip
    \textbf{(1.3.6a)}
    The following is a second-dimensional representation of the concept. Note that the only 
    point not reached by constructing lines this way is the "north pole", since that point 
    cannot be represented by a line that intersects the sphere at two places using our 
    current modeling.
    
    \includegraphics[width=5cm]{6a.png}
    
    \bigskip
    \textbf{(1.3.6b)}
    The line from $(0,0,1)$ to $(u,v,0)$ can be parametrized by 
    $(1-t)(0,0,1) + t(u,v,0) = (tu, tv, 1-t)$. 
    
    \bigskip
    \textbf{(1.3.6c)}
    Substituing $x = tu, y = tv, z =1 - t$ into $x^2 + y^2 + z^2 - 1 = 0$ yields
    \[
        0 = (tu)^2 + (tv)^2 + (1-t)^2 - 1 = (u^2 + v^2 + 1)t^2 - 2t = t((u^2 + v^2 + 1)t - 2).
    \]
    Note that, since $t = 0$ corresponds to the point $(0,0,1)$, the other point of 
    intersection must be given by $t = \frac{2}{u^2+v^2+1}$. Thus we have that 
    \[
        x = tu = \frac{2u}{u^2 + v^2 + 1}, y = tv = \frac{2v}{u^2 + v^2 + 1}, z = 1 - t = \frac{u^2 + v^2 - 1}{u^2 + v^2 + 1}.
    \]
\end{exercise}

%%%%%%%%%%%%%%%%%%%%%%%%%%%%%%%%%%%%%%%%

\begin{exercise}{1.3.7}
    Working in $\R^n$ means that the "north pole" is given by $(0,\ldots,0,1)$. Next, we 
    need to define the line through the north pole and a point $(u_1, \ldots, u_{n-1}, 0)$
    in the hyperplane $x_n = 0$. This can be defined by 
    \[
        (x_1, \ldots, x_n) = (1-t)(0, \ldots, 0, 1) + t(u_1, \ldots, u_{n-1}, 0) = (tu_1, \ldots, tu_{n-1}, 1 - t).
    \]
    We can observe that this line meets $x_1^2 + \ldots + x_n^2 - 1 = 0$ where 
    \[
        0 = (tu_1)^2 + \ldots + (tu_{n-1}^2) + (1 - t)^2 - 1 = (u_1^2 + \ldots + u_{n-1}^2 + 1)t^2 - 2t = t((u_1^2 + \ldots + u_{n-1}^2 + 1)t - 2).
    \]
    We already know that $t = 0$ is the north pole, so the other point of intersection 
    must be given by 
    \[
        t = \frac{2}{u_1^2 + \ldots + u_{n - 1}^2 + 1}
    \]
    which gives the parametrization
    \[
        x_i = tu_i = \frac{2u_i}{u_1^2 + \ldots + u_{n - 1}^2 + 1}
    \]
    for all $1 \leq i \leq n$.
\end{exercise}

%%%%%%%%%%%%%%%%%%%%%%%%%%%%%%%%%%%%%%%%

\begin{exercise}{1.3.11}

    \bigskip
    \textbf{(1.3.11a)}
    Using 1.3.8, we can change the variables $y \rightarrow x, x \rightarrow z$ to 
    see that $x^2 = cz^2 - z^3$ is parametrized by $z = c - t^2, x = t(c - t^2)$.
    
    \bigskip
    \textbf{(1.3.11b)}
    Using $c = y^2$ for some fixed $y$, it follows that $x^2 = y^2z^2 - z^3$ is 
    parametrized by $z = y^2 - t^2, x = t(y^2 - t^2)$. Finally, we replace $y$ 
    with a parameter $u$ to get the parametrization $x = t(u^2 - t^2), y = u, z = u^2 - t^2$
    of $x^2 = y^2z^2 - z^3$, which is the surface $V(x^2 - y^2z^2 + z^3)$ in $\R^3$.
    
    \bigskip
    \textbf{(1.3.11c)}
    Let $(x,y,z) \in V(x^2 - y^2z^2 + z^3)$, and define $c = y^2$. We then have that 
    $(x,z) \in V(x^2 - cz^2 - z^3)$. Using (1.3.8c) with the change of variables 
    $y \rightarrow x, x \rightarrow z$, there exists a $t$ such that 
    $z = c - t^2, x = t(c - t^2)$. Now, let $u = y$ so that $c = y^2 = u^2$. Then 
    $x = t(u^2-t^2), y = u, z = u^2 - t^2$ as desired.
\end{exercise}

%%%%%%%%%%%%%%%%%%%%%%%%%%%%%%%%%%%%%%%%

\begin{exercise}{1.3.13}
    Using method 1, we have that $x = 1 + u - v, y = u + 2v, z = -1 - u + v$ satisfies 
    $ax + by +cz = d$ if and only if 
    \[
        d = a(1 + u - v) + b(u + 2v) + c(-1 - u + v) = (a + b - c)u + (-a + 2b + c)v + (a - c).
    \]
    We can rewrite this as the following system of equations
    \[
        \begin{cases}
            a + b - c = 0\\
            -a + 2b + c = 0\\
            a - c = d.
        \end{cases}
    \]
    When solved, this system yields $a = c = 1, b = d = 0$, so the equation is $x + z = 0$.
\end{exercise}

%%%%%%%%%%%%%%%%%%%%%%%%%%%%%%%%%%%%%%%%
\newpage
\begin{exercise}{1.3.14}

    \bigskip
    \textbf{(1.3.14a)}
    Let $P, Q \in \R^2$. Then the line connecting $P$ and $Q$ is parametrized by 
    \[
        P + t(Q - P) = (1 - t)P + tQ
    \]
    for $t \in \R$. Note that $t = 0$ yields $P$, $t = 1$ yields $Q$, and $0 \leq t \leq 1$
    yields the line segment joining the two. Thus we have that 
    \[
        (1-t)P + tQ \in C
    \]
    for $0 \leq t \leq 1$.
    
    \bigskip
    \textbf{(1.3.14b)}
    We wish to show that, for a convex set $C$ containing $P_1, \ldots, P_n$, we have 
    $\sum_{i = 1}^nt_iP_i \in C$ when $t_1, \ldots, t_n \geq 0$ and that $\sum_{i = 1}^n t_i = 1$. 
    We proceed by induction on $n$. The base case of $n = 1$ is true by observation, 
    so we assume that the assertion is true for $n$. Next, we consider 
    $P_1, \ldots, P_{n + 1} \in C$ with $t_1, \ldots, t_{n+1} \geq 0$ and 
    $\sum_{i = 1}^{n + 1} t_i = 1$. This implies that $t_{n + 1} \leq 1$ and yields two 
    possible cases:
    
    \textit{Case 1}: $t_{n+1} = 1$. Then $t_1 = \ldots = t_n = 0$, so 
    $\sum_{i = 1}^{n+1}t_iP_i = P_{n + 1} \in C$.
    
    \textit{Case 2}: $t_{n+1} < 1$. Then $1 - t_{n+1} > 0$ and we fix 
    $u = \frac{t_i}{1-t_{n+1}}$ for $1 \leq i \leq n$. Then $u_i \geq 0$. Thus 
    $\sum_{i = 1}^nt_i = 1-t_{n+1}$ implies $\sum_{i = 1}^{nu}u_i = 1$. Using the 
    inductive hypothesis, $P = \sum_{i = 1}^nu_iP_i \in C$. We can then use (1.3.14a)
    to show that $C$ contains the point as follows:
    \begin{align*}
        (1-t_{n+1})P + t_{n+1}P_{n+1}   &= (1-t_{n+1})\sum_{i = 1}^nu_iP_i + t_{n+1}P_{n+1}\\
                                        &= (1-t_{n+1})\sum_{i = 1}^n\frac{t_i}{1-t_{n+1}}P_i + t_{n+1}P_{n+1}\\
                                        &= \sum_{i = 1}^nt_iP_i + t_{n+1}P_{n+1}\\
                                        &= \sum_{i = 1}^{n+1}t_iP_i
    \end{align*}
    as desired.
\end{exercise}

%%%%%%%%%%%%%%%%%%%%%%%%%%%%%%%%%%%%%%%%

%---------------------------------
% Don't change anything below here
%---------------------------------


\end{document}
