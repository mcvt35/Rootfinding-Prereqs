\documentclass[12pt,oneside]{article}

% This package simply sets the margins to be 1 inch.
\usepackage[margin=1in]{geometry}

% These packages include nice commands from AMS-LaTeX
\usepackage{amssymb,amsmath,amsthm,graphicx}


% Define an environment for exercises.
\newenvironment{exercise}[1]{\vspace{.1in}\noindent\textbf{Exercise #1 \hspace{.05em}}}{}

% define shortcut commands for commonly used symbols
\newcommand{\R}{\mathbb{R}}
\newcommand{\C}{\mathbb{C}}
\newcommand{\Z}{\mathbb{Z}}
\newcommand{\Q}{\mathbb{Q}}
\newcommand{\N}{\mathbb{N}}


%%%%%%%%%%%%%%%%%%%%%%%%%%%%%%%%%%%%%%%%%%

\begin{document}

% If you use Overleaf, the name of the project will be determined by
% what you enter as the document title.
\title{Math Homework Template}

\begin{flushright}
\textsc{Marcelo Leszynski}  \\
Rootfinding Research Prerequisites\\
07/13/20
\end{flushright}

\begin{center}
\textsf{Assignment 2.1 } \\
\textsf{Exercises: 1b, 2b, 3b, 4, 5 }
\end{center}

%%%%%%%%%%%%%%%%%%%%%%%%%%%%%%%%%%%%%%%%

\begin{exercise}{2.1.1b}
    Dividing the basis $x^3-x^2+x$ into $x^5-4x+1$ using the division algorithm
    yields
    \[
        x^5-4x+1=(x^2+x)(x^3-x^2+x)+(-x^2-4x+1)
    \]
    which has a nonzero remainder term ($-x^2-4x+1$). Therefore, 
    $f(x) = x^5-4x+1 \not \in I = \langle x^3-x^2+x \rangle$.
\end{exercise}

%%%%%%%%%%%%%%%%%%%%%%%%%%%%%%%%%%%%%%%%

\begin{exercise}{2.1.2b}
    We begin by constructing the coefficient matrix (in this case, it 
    is not an augmented matrix since each equation is set to zero). We
    then row reduce the coefficient matrix.
    \[
        \begin{bmatrix}
            1 & 1 & -1 & -1\\
            1 & -1 & 1 & 0
        \end{bmatrix}
        \rightarrow
        \begin{bmatrix}
            1 & 0 & 0 & -\frac{1}{2}\\
            0 & 1 & -1 & -\frac{1}{2}
        \end{bmatrix}
    \]
    The resulting matrix shows that $x_3, x_4$ are free variables. We will 
    parameterize them by setting $x_3 = t, x_4 = u$ which yields
    \begin{align*}
        x_1 &= \frac{1}{2}u\\
        x_2 &= t + \frac{1}{2}u\\
        x_3 &= t\\
        x_4 &= u.
    \end{align*}
\end{exercise}

%%%%%%%%%%%%%%%%%%%%%%%%%%%%%%%%%%%%%%%%

\begin{exercise}{2.1.3b}
    We begin by rewriting the given parameterization in the following form:
    \begin{align*}
        2t-5u-x_1   &=0\\
        t+2u-x_2    &=0\\
        -t+u-x_3    &=0\\
        t+3u-x_4    &=0
    \end{align*}
    Using this parameterization, we construct the following coefficient 
    matrix and row reduce it.
    \[
        \begin{bmatrix}
            2&-5&-1&0&0&0\\
            1&2&0&-1&0&0\\
            -1&1&0&0&-1&0\\
            1&3&0&0&0&-1
        \end{bmatrix}
        \rightarrow
        \begin{bmatrix}
            1&0&0&0&\frac{3}{4}&-\frac{1}{4}\\
            0&1&0&0&-\frac{1}{4}&-\frac{1}{4}\\
            0&0&1&0&\frac{11}{4}&\frac{3}{4}\\
            0&0&0&1&\frac{1}{4}&-\frac{3}{4}
        \end{bmatrix}
    \]
    This results in the equations which define the affine variety:
    \begin{align*}
        x_1 + \frac{11}{4}x_3+\frac{3}{4}x_4    &=0\\
        x_2+\frac{1}{4}x_3-\frac{3}{4}x_4       &=0
    \end{align*}
\end{exercise}

%%%%%%%%%%%%%%%%%%%%%%%%%%%%%%%%%%%%%%%%

\begin{exercise}{2.1.4}

    \bigskip
    \textbf{(2.1.4a):}
    To show that $I$ is an ideal in the ring $R$, we first observe that 
    $R$ contains the zero polynomial, which implies that $I$ does as well 
    in the from
    \[
        0 = 0 \cdot x_1 + 0 \cdot x_2 + \ldots
    \]
    It is also worth noting that $I$ is nonempty since it contains $x_1, x_2,$ etc.

    Next, let $p,q \in I$ be arbitrary polynomials in $I$. This implies that
    \begin{align*}
        p   &= f_1 \cdot x_1 + f_2 \cdot x_2 + \ldots + 0 \cdot x_i + 0 \cdot x_{i+1} + \ldots\\
        q   &= g_1 \cdot x_1 + g_2 \cdot x_2 + \ldots + 0 \cdot x_i + 0 \cdot x_{i+1} + \ldots\\
        p+q &= f_1 \cdot x_1 + g_1 \cdot x_1 + f_2 \cdot x_2 + g_2 \cdot x_2 + \ldots + 0 \cdot x_i + 0 \cdot x_{i+1} + \ldots\\
            &= (f_1 + g_1) \cdot x_1 + (f_2 + g_2) \cdot x_2 + \ldots + 0 \cdot x_i + 0 \cdot x_{i+1} + \ldots
    \end{align*}
    which shows that any arbitrary addition of polynomials in $I$ is also 
    contained in $I$.

    Finally, let $t$ be a polynomial in $I$ and let $r \in R$. Then 
    $rt$ is given by 
    \begin{align*}
        rt  &= r(f_1 \cdot x_1 + f_2 \cdot x_2 + \ldots + 0 \cdot x_i + 0 \cdot x_{i+1} + \ldots\\
            &= (rf_1) \cdot x_1 + (rf_2) \cdot x_2 + \ldots + 0 \cdot x_i + 0 \cdot x_{i+1} + \ldots\\
            &= g_1 \cdot x_1 + g_2 \cdot x_2 + \ldots + 0 \cdot x_i + 0 \cdot x_{i+1} + \ldots
    \end{align*}
    with $g_i = rf_i \in R$. Therefore, $I$ is an ideal in the ring $R$.

    \bigskip
    \textbf{(2.1.4b):}
    Suppose, by way of contradiction, that $I$ has a finite generating set 
    $I = \langle g_1,\ldots,g_n \rangle$. Since each $g_i$ contains a finite 
    number of variables, there exists some variable $x_\emptyset$ which does 
    not appear in any $g_i$. This implies that the constant term associated with
    the $x_\emptyset$ variable in each $g_i$ is equal to zero. This follows for 
    all nonzero $f \in \langle g_1, \ldots, g_n \rangle$. This would mean that 
    $x_\emptyset \not \in \langle g_1, \ldots, g_n \rangle = I = \langle x_1, x_2, \ldots \rangle$.
    This is a contradiction, so there exists no finite generating set for the 
    given ideal.

\end{exercise}

%%%%%%%%%%%%%%%%%%%%%%%%%%%%%%%%%%%%%%%%

\begin{exercise}{2.1.5}

    \bigskip
    \textbf{(2.1.5a):}
    \begin{proof}
        We proceed with induction on $m \geq 0$. 

        \underline{Base Case}: When $m=0$, the only polynomial of the given 
        form with degree $\leq 0$ is $x^0y^0=1$ and $\frac{1}{2}(x+1)(0+2) = 1$, 
        so the base case holds.

        \underline{Inductive Hypothesis}: The number of distinct monomials $x^ay^b$ 
        of total degree $leq n$ in $k[x,y]$ is equal to $(m+1)(m+2)/2$ when $n\geq 0$.
        
        \underline{Inductive Step}: Notice that the set of all polynomials of 
        degree $\leq n+1$ can be described as 
        \[
            \{x^ay^b | a+b \leq n+1\} = \{x^ay^b | a+b \leq n\} \cup \{x^{m+1}, \ldots, y^{m+1}\}
        \]
        Note that this union is disjoint, and that the set on the right hand side 
        of the union consists of $n+2$ polynomials as described in our inductive hypothesis.
        This allows us to conclude that the number of monomials of total degree 
        $\leq n+1$ is given by
        \begin{align*}
            \frac{1}{2}(m+1)(m+2) + (m+2)   &= \frac{1}{2}(2(\frac{1}{2}(m+1)(m+2)+(m+2)))\\
                                            &= \frac{1}{2}((m+1)(m+2) + 2(m+2))\\
                                            &= \frac{1}{2}(m^2+5m+6)\\
                                            &= \frac{1}{2}(m+2)(m+3)
        \end{align*}
        as needed. 
    \end{proof}
    
    \bigskip
    \textbf{(2.1.5b):} Let $f(t), g(t)$ be polynomials of degree $\leq n$ in $t$. 
    Suppose, by way of contradiction, that for $m$ large enough, the "monomials"
    $[f(t)]^a[g(t)]^b$ with $a+b \leq m$ are linearly independent. Then we have 
    that $[f(t)]^a[g(t)]^b$ has degree $\leq n(a+b) \leq nm$. This implies that 
    $[f(t)]^a[g(t)]^b$ lies in the vector space of polynomials of degree $leq nm$ in $t$,
    which is of dimension $nm+1$. Exercise 2.1.5a shows that there are $\frac{1}{2}(m+1)(m+2)$ 
    choices for exponents $a$ and $b$ with $a+b \leq m$. Note that, as $m$ increases, 
    we have that 
    \[
        \frac{1}{2}(m+1)(m+2) > nm+1.    
    \]
    This is a contradiction, because this implies that $[f(t)]^a[g(t)]^b$ are 
    linearly dependent. This is because $[f(t)]^a[g(t)]^b$ is in a vector space 
    of dimension $nm+1$. We therefore conclude that the given "monomials" are 
    linearly dependent with a large enough $m$ value. 

    \bigskip
    \textbf{(2.1.5c):} (Note: understanding this problem required lots of help from 
    the internet. Much of what was being asked went over my head without additional explanation). 
    Let the nontrivial relation described in Exercise 2.1.5b be given as 
    \[
        \sum_{a+b\leq m}c_{a,b}f(t)^ag(t)^b=0.
    \]
    We then have that $F = \sum_{a+b\leq m}c_{a,b}x^ay^b \in k[x,y]$ is a nonzero 
    polynomial. Furthermore, we have that $F(f(t),g(t)) = 0$ is the zero polynomial in $t$.
    It follows that for all $t \in k, (f(t),g(t)) \in V(F)$. Therefore, $V(F)$ contains 
    the curve in $k^2$ parameterized by $t \to (f(t),g(t))$.

    \bigskip
    \textbf{(2.1.5d):}
    \begin{proof}
        We begin by proving a more generalized version of the proposition in 
        Exercise 2.1.5a by induction. We will show that the number of monomials 
        in the form $x^ay^bz^c$ of total degree $\leq m$ is given by 
        $\frac{1}{6}(m+1)(m+2)(m+3)$, i.e., the binomial coefficient $\binom{m+3}{3}$.

        \underline{Base Case}: When $m = 0$, the only possible polynomial in the given 
        form of degree $leq m$ is $x^0y^0z^0 = 1$. Note that $\frac{1}{6}(0+1)(0+2)(0+3) = 1$.
        Therefore, the base case holds.

        \underline{Inductive Hypothesis}: The number of distinct monomials $x^ay^bz^c$ 
        of total degree $leq n$ in $k[x,y,z]$ is equal to $\frac{1}{6}(m+1)(m+2)(m+3)$ when $n\geq 0$.

        \underline{Inductive Step}: Notice that the set of all polynomials of the 
        given form with degree $leq n+1$ can be described as
        \[
            \{x^ay^bc^z | a+b+c \leq n+1\} = \{x^ay^bz^c | a+b+c \leq n\} \cup \{x^ay^bz^c | a+b+c = m+1\}
        \]
        It is worth noting that the two sets on the right hand side are disjoint, and 
        the sum of their cardinalities is equal to the cardinality of the set on the left hand side.
        Using our inductive hypothesis together with Exercise 2.1.5a, we can see that 
        the number of monomials of total degree $\leq n+1$ is given by 
        \[
            \frac{1}{6}(m+1)(m+2)(m+3) + \frac{1}{2}(m+2)(m+3) = \frac{1}{6}(m+2)(m+3)((m+1)+3) = \frac{1}{6}(m+2)(m+3)(m+4).    
        \]

        For the next part of our proof, we let $f(t,u),g(t,u),h(t,u)$ be 
        polynomials of total degree $\leq d$. For a given integer $m$ such 
        that $a+b+c \leq m$, we see that the polynomials $[f(t,u)]^a[g(t,u)]^b[h(t,u)]^z$ 
        have a total degree of $d(a+b+c) \leq dm$. We concluded above 
        that there are only  $\frac{1}{6}(m+1)(m+2)(m+3)$ choices for $a,b,c$ with 
        $a+b+c \leq m$. Thus, these polynomials are contained in a vector space of dimension
        $\frac{1}{2}(dm+1)(dm+2)$ as shown in Exercise 2.1.5a. Therefore, these 
        polynomials must be linearly dependent when $m$ is chosen to be sufficiently 
        large such that 
        \[
            \frac{1}{6}(m+1)(m+2)(m+3) > \frac{1}{2}(dm+1)(dm+2).
        \]

        We now write this nontrivial dependence as 
        \[
            \sum_{a+b+c\leq m}c_{a,b,c}x^ay^bz^c \in k[x,y,z]    
        \]
        and note that, when this dependece is equal to zero, then there is 
        a nonzero polynomial 
        \[
            F = \sum_{a+b+c\leq m}c_{a,b,c}x^ay^bz^c \in k[x,y,z]    
        \]
        that has the property that $F(f(t,u),g(t,u),h(t,u)) = 0 \in k[t,u]$.
        This implies that $V(F)$ contains the surface in $k^3$ parameterized 
        by $(t,u) \to (f(t,u),g(t,u),h(t,u))$.
    \end{proof}
\end{exercise}

%%%%%%%%%%%%%%%%%%%%%%%%%%%%%%%%%%%%%%%%

%---------------------------------
% Don't change anything below here
%---------------------------------


\end{document}