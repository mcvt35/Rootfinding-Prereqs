\documentclass[12pt,oneside]{article}

% This package simply sets the margins to be 1 inch.
\usepackage[margin=1in]{geometry}

% These packages include nice commands from AMS-LaTeX
\usepackage{amssymb,amsmath,amsthm,graphicx}

% Define an environment for exercises.
\newenvironment{exercise}[1]{\vspace{.1in}\noindent\textbf{Exercise #1 \hspace{.05em}}}{}

% define shortcut commands for commonly used symbols
\newcommand{\R}{\mathbb{R}}
\newcommand{\C}{\mathbb{C}}
\newcommand{\Z}{\mathbb{Z}}
\newcommand{\Q}{\mathbb{Q}}
\newcommand{\N}{\mathbb{N}}

%%%%%%%%%%%%%%%%%%%%%%%%%%%%%%%%%%%%%%%%%%

\begin{document}

% If you use Overleaf, the name of the project will be determined by
% what you enter as the document title.
\title{Math Homework Template}

\begin{flushright}
\textsc{Marcelo Leszynski}  \\
Rootfinding Research Prerequisites\\
07/30/20
\end{flushright}

\begin{center}
\textsf{Assignment 2.4 } \\
\textsf{Exercises: 1, 3, 4 }
\end{center}

%%%%%%%%%%%%%%%%%%%%%%%%%%%%%%%%%%%%%%%%

\begin{exercise}{2.4.1}
    \begin{proof}
        Let $G = \langle x^\alpha | x^\alpha \text{ can be found in some }f \in I\rangle$
        be a monomial ideal. If $I$ is a monomial ideal, then $I=G$ by Lemma 3 
        and Corollary 4. 

        To show that $I=G$, we examine $g \in G$ which can be expressed as the 
        sum of terms $cx^\alpha$ where $x^\alpha$ is a monomial contained in some 
        $f \in I$. By construction, we then have that $x^\alpha \in I$, so it follows 
        that $g \in I$ by the definition of an ideal. To show the other inclusion, 
        suppose that $g \in I$. This means that for all terms $cx^\alpha$ contained 
        in $g$, $x^\alpha \in G$ by the construction of $G$. It then follows that 
        $g \in G$ since $G$ is an ideal. 
    \end{proof}
\end{exercise}

%%%%%%%%%%%%%%%%%%%%%%%%%%%%%%%%%%%%%%%%

\begin{exercise}{2.4.3}

    \bigskip
    \textbf{(2.4.3a):}

    \includegraphics[scale=0.25]{2-4-3a.png}

    \bigskip
    \textbf{(2.4.3b):}
    \begin{align*}
        &1,x,x^2,x^3,x^4,x^5\\
        &y,xy,x^2y,x^3y,x^4y,x^5y\\
        &y^2,xy^2,x^2y^2,x^3y^2,x^4y^2,x^5y^2\\
        &y^3,xy^3\\
        &y^4,xy^4\\
        &y^5,xy^5\\
        &y^6,xy^6\\
        &y^\alpha, \alpha \geq 7
    \end{align*}
\end{exercise}

%%%%%%%%%%%%%%%%%%%%%%%%%%%%%%%%%%%%%%%%

\begin{exercise}{2.4.4}

    \bigskip
    \textbf{(2.4.4a):}
    Let $J = \langle x^\alpha | x^\alpha y^\beta\in I \rangle$ for some $\beta \geq 0$.
    Since $x^3y^6 \in I$, we have that $J = \langle x^3 \rangle$ with $\beta = 6$.
    This implies that $J_0=J_1=J_2=J_3=\langle x^6 \rangle$ and $J_4=J_5=\langle x^5 \rangle$.
    Then, by Theorem 5, we have that $I=\langle x^3y^6,x^6,x^6y,x^6y^2,x^6y^3,x^5y^4,x^5y^5\rangle$.

    \bigskip
    \textbf{(2.4.4b):}
    Removing all terms in the previous basis that can be divided by other 
    distinct terms results in the basis $I = \langle x^3y^6,x^6,x^5y^4 \rangle$.

\end{exercise}

%%%%%%%%%%%%%%%%%%%%%%%%%%%%%%%%%%%%%%%%

%---------------------------------
% Don't change anything below here
%---------------------------------

\end{document}