\documentclass[12pt,oneside]{article}

% This package simply sets the margins to be 1 inch.
\usepackage[margin=1in]{geometry}

% These packages include nice commands from AMS-LaTeX
\usepackage{amssymb,amsmath,amsthm,graphicx}

% Define an environment for exercises.
\newenvironment{exercise}[1]{\vspace{.1in}\noindent\textbf{Exercise #1 \hspace{.05em}}}{}

% define shortcut commands for commonly used symbols
\newcommand{\R}{\mathbb{R}}
\newcommand{\C}{\mathbb{C}}
\newcommand{\Z}{\mathbb{Z}}
\newcommand{\Q}{\mathbb{Q}}
\newcommand{\N}{\mathbb{N}}

%%%%%%%%%%%%%%%%%%%%%%%%%%%%%%%%%%%%%%%%%%

\begin{document}

% If you use Overleaf, the name of the project will be determined by
% what you enter as the document title.
\title{Math Homework Template}

\begin{flushright}
\textsc{Marcelo Leszynski}  \\
Rootfinding Research Prerequisites\\
07/30/20
\end{flushright}

\begin{center}
\textsf{Assignment 2.8 } \\
\textsf{Exercises: 1-5 }
\end{center}

%%%%%%%%%%%%%%%%%%%%%%%%%%%%%%%%%%%%%%%%

\begin{exercise}{2.8.1}
    It is true that $f \in I$.
    \begin{proof}
        We begin by computing a Groebner basis $G$ in lex order 
        \[
            G = \{y^5-z^3,-y^2+xz,xy^3-z^2,x^2y-z,x^3-y\}.
        \]
        It suffices to show that the remainder of $f$ when divided by the 
        terms in $G$ is zero. Using the division algorithm, we have 
        \[
            f=1\cdot(y^5-z^3)+0\cdot(-y^2+xz)+1\cdot(xy^3-z^2)+0\cdot(x^2y-z)+0\cdot(x^3-y)+0.
        \]
        Therefore, $f \in I$.
    \end{proof}
\end{exercise}

%%%%%%%%%%%%%%%%%%%%%%%%%%%%%%%%%%%%%%%%

\begin{exercise}{2.8.2}
    It is not true that $f \in I$.
    \begin{proof}
        We begin by computing a Groebner basis $G$ in lex order
        \[
            G = \{2z^2+z,y-z,xz-z\}.
        \]
        It suffices to show that the remainder of $f$ when divided by the 
        terms in $G$ is nonzero. Using the division algorithm, we have 
        \[
            f=-1\cdot(2z^2+z)+(-2y-2z)\cdot(y-z)+(x^2+x+1)\cdot(xz-z)+2z.
        \]
        Therefore, $f \not \in I$.
    \end{proof}
\end{exercise}

%%%%%%%%%%%%%%%%%%%%%%%%%%%%%%%%%%%%%%%%

\begin{exercise}{2.8.3}
    We begin by computing a Groebner basis $G$ in lex order
    \[
        G = \{g_1,g_2,g_3\} = \{40z^2-8z-23,3y+z-1,2x-1\}.
    \]
    Note that $g_3$ gives $x=\frac{1}{2}$ and that $g_1$ is a quadratic in terms of $z$,
    which we can solve to get $z=\frac{2\pm 3\sqrt{26}}{20}$. Using these values to solve 
    for $y$ in $g_2$ gives the roots
    \[
        (x,y,z) = \biggr(\frac{1}{2},\frac{6-\sqrt{26}}{20},\frac{2+3\sqrt{26}}{20}\biggr),\biggr(\frac{1}{2},\frac{6+\sqrt{26}}{20},\frac{2-3\sqrt{26}}{20}\biggr).
    \]
\end{exercise}

%%%%%%%%%%%%%%%%%%%%%%%%%%%%%%%%%%%%%%%%

\newpage
\begin{exercise}{2.8.4}
    Computing a Groebner basis $G$ using lex order for the ideal 
    \[
        \langle x^2y-z^3,2xy-4z-1,-y^2+z,x^3-4yz\rangle
    \]
    yields $G=\{1\}$. Therefore, 
    \[
        V(x^2y-z^3,2xy-4z-1,-y^2+z,x^3-4yz) = \emptyset.
    \]
\end{exercise}

%%%%%%%%%%%%%%%%%%%%%%%%%%%%%%%%%%%%%%%%

\begin{exercise}{2.8.5}

    \bigskip
    \textbf{(2.8.5a):}
    To find all critical points of the function $f(x,y)$, we begin by calculating the 
    partial derivatives 
    \begin{align*}
        f_x(x,y) &= 4x^3+4xy^2-8x-3\\
        f_y(x,y) &= 4y^3+4x^2y-8y-3.
    \end{align*}
    Next, we compute a Groebner basis $G$ for $\langle f_x(x,y), f_y(x,y)\rangle$ and have 
    \[
        G=\{g_1,g_2\}=\{x-y,8y^3-8y-3\}.
    \]
    Note that $g_2$ can be factored into $(2y+1)(4y^2-2y+3)$. This, along with the 
    fact that $x=y$ is given by $g_!$, yields three critical points 
    \[
        \biggr(-\frac{1}{2},-\frac{1}{2}\biggr),\biggr(\frac{1}{4}(1-\sqrt{13}),\frac{1}{4}(1-\sqrt{13})\biggr),\biggr(\frac{1}{4}(1+\sqrt{13}),\frac{1}{4}(1+\sqrt{13})\biggr).
    \]
    
    \bigskip
    \textbf{(2.8.5b):}
    To determine whether the critical points are local minima/maxima or saddle points, we
    use the second derivative test and begin by calculating the second-order partial derivatives 
    \begin{align*}
        f_{xx}(x,y) &= 12x^2+4y^2-8\\
        f_{xy}(x,y) &= 8xy\\
        f_{yy}(x,y) &= 12y^2+4x^2-8.
    \end{align*}
    Next, we evaluate $D=f_{xx}f_{yy}-f_{xy}^2$ at the critical points. Remembering that 
    $x=y$ was given to us by our Groebner basis, we have that 
    \begin{align*}
        D(x,x)  &= 192x^4-256x^2+64 && \text{so}\\
        D\biggr(-\frac{1}{2},-\frac{1}{2}\biggr)    &= 12 > 0   && \text{is undetermined,}\\
        D\biggr(\frac{1}{4}(1+\sqrt{13}),\frac{1}{4}(1+\sqrt{13})\biggr)    &=26+10\sqrt{13} > 0 &&\text{is undetermined,}\\
        D\biggr(\frac{1}{4}(1-\sqrt{13}),\frac{1}{4}(1-\sqrt{13})\biggr)    &=26-10\sqrt{13} < 0 &&\text{is a saddle point.}
    \end{align*}
    From this, we conclude that $(\frac{1}{4}(1-\sqrt{13}),\frac{1}{4}(1-\sqrt{13}))$ is 
    a saddle point. The status of the remaining two critical points can be ascertained as 
    follows:
    \begin{align*}
        f_{xx}\biggr(-\frac{1}{2},-\frac{1}{2}\biggr)   &= -4 < 0\\
        f_{xx}\biggr(\frac{1}{4}(1+\sqrt{13}),\frac{1}{4}(1+\sqrt{13})\biggr)   &= 6+2\sqrt{13} >0.
    \end{align*}
    Thus we see that $(-\frac{1}{2},-\frac{1}{2})$ is a local maximum and that 
    $(\frac{1}{4}(1+\sqrt{13}),\frac{1}{4}(1+\sqrt{13}))$ is a local minimum.
\end{exercise}

%%%%%%%%%%%%%%%%%%%%%%%%%%%%%%%%%%%%%%%%
%---------------------------------
% Don't change anything below here
%---------------------------------

\end{document}