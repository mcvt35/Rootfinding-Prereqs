\documentclass[12pt,oneside]{article}

% This package simply sets the margins to be 1 inch.
\usepackage[margin=1in]{geometry}

% These packages include nice commands from AMS-LaTeX
\usepackage{amssymb,amsmath,amsthm,graphicx}

% Define an environment for exercises.
\newenvironment{exercise}[1]{\vspace{.1in}\noindent\textbf{Exercise #1 \hspace{.05em}}}{}

% define shortcut commands for commonly used symbols
\newcommand{\R}{\mathbb{R}}
\newcommand{\C}{\mathbb{C}}
\newcommand{\Z}{\mathbb{Z}}
\newcommand{\Q}{\mathbb{Q}} 
\newcommand{\N}{\mathbb{N}} 

%%%%%%%%%%%%%%%%%%%%%%%%%%%%%%%%%%%%%%%%%%

\begin{document}

% If you use Overleaf, the name of the project will be determined by
% what you enter as the document title.
\title{Math Homework Template}

\begin{flushright}
\textsc{Marcelo Leszynski}  \\
Rootfinding Research Prerequisites\\
10/07/20
\end{flushright}

\begin{center}
\textsf{Assignment 5.2 } \\
\textsf{Exercises: 1, 2, 3, 5, 8, 13, 16 }
\end{center}

%%%%%%%%%%%%%%%%%%%%%%%%%%%%%%%%%%%%%%%%

\begin{exercise}{5.2.1}
    To determine wheter $f \equiv g \mod{I}$, first compute 
    a Grobner basis $G$ for $I = \langle f_1,\cdots,f_s\rangle$, 
    then check to see if $\overline{f-g}^G = 0$. If yes, then 
    $f\equiv g \mod{I}$. Else, $f \not \equiv g \mod{I}$.
\end{exercise}

%%%%%%%%%%%%%%%%%%%%%%%%%%%%%%%%%%%%%%%%

\begin{exercise}{5.2.2}
    \begin{proof}
        We begin by defining a map $\Phi$ mapping $\phi$ to the 
        equivalence class of $[f]$ congruent modulo $\mathbf{I}(V)$.
        In other words, we have that $f$ represents $\phi$. Likewise,
        we define a map $\Psi$ that maps the equivalence class $[g]$ 
        congruent module $\mathbf{I}(V)$ to the polynomial function 
        $\psi: V \to k$ represented by $g$. Using Proposition 5.1.2, 
        these two maps are well defined. Thus, we suppose that $\phi$ 
        is represented by $f$ and have that 
        \[
            \Psi(\Phi(\phi)) = \Psi([f])   
        \]
        is equal to the polynomial function represented by $f = \phi$,
        and that
        \[
            \Phi(\Psi([f])) = [f]    
        \]
        is equal to $\Psi$ of the polynomial function represented by $f$.

        It follows that $\Phi$ and $\Psi$ are inverses of each other, 
        so the distinct polynomial functions $\phi$ are in one-to-one
        correspondence with the equivalence classes of polynomials 
        under congruence module $\mathbf{I}(V)$ as desired.
    \end{proof}
\end{exercise}

%%%%%%%%%%%%%%%%%%%%%%%%%%%%%%%%%%%%%%%%

\begin{exercise}{5.2.3}
    \begin{proof}
        Let $I$ be an ideal such that $I \in k[x_1,\ldots,x_n]$.
        Let $[f],[g],[h] \in k[x_1,\ldots,x_n]/I$. Using the 
        fact that $k[x_1,\ldots,x_n]$ is a commutative ring, 
        we have that 
        \begin{align*}
            &([f]+[g])+[h] = [(f+g)+h]=[f+(g+h)]=[f]+([g]+[h]) && \text{associativity (+)}\\
            &[f]+[g] = [f+g] = [g+f] = [g]+[f]  && \text{commutativity (+)}\\
            &[f]\cdot[g] = [f\cdot g] = [g \cdot f] = [g] \cdot [f]  && \text{commutativity (*)}\\
            &[f]\cdot([g]+[h]) = [f \cdot (g+h)] = [f] \cdot [g] + [f] \cdot [h]    && \text{distributivity}\\
            &[f] + [0] = [f+0] = [f]    && \text{additive ident.}\\
            &[f] \cdot [1] = [f \cdot 1] = [f]  && \text{multiplicative ident.}\\
            &[f] + [-f] = [f+(-f)]=[0]  && \text{additive inverse}
        \end{align*}
    \end{proof}
\end{exercise}

%%%%%%%%%%%%%%%%%%%%%%%%%%%%%%%%%%%%%%%%

\begin{exercise}{5.2.5}

    \bigskip
    \textbf{(5.2.5a):}
    Using the division algorithm in $\R[x]$, we have that 
    $f = q(x^2+1)+r$ for some $q,r \in \R[x]$ where $r=0$ or $\deg{(r)} < 2$ 
    is unique. Thus we can write $r = ax+b$ for some $a,b \in \R$.
    Since $f-r = q(x^2+1) \in I= \langle x^2+1 \rangle$, $r$ satisfies
    the desired conditions of congruence.

    \bigskip
    \textbf{(5.2.5b):}
    We define addition and multiplication in $\R[x]/\langle x^2+1 \rangle$ 
    as follows:
    \[
        [ax+b]+[cx+d]       = [(a+c)x+(b+d)],
    \]
    \[
        [ax+b]\cdot[cx+d]   = [(ax+b)(cx+d)] = [acx^2+(ad+bc)x+bd] = [(ad+bc)x+(bd-ac)].
    \]

    \bigskip
    \textbf{(5.2.5c):}
    We can observe that $\R[x]/\langle x^2+1 \rangle$ is ring 
    isomorphic to $\C$ by noting that $[x^2+1]=[0]$ in 
    $\R[x]\langle x^2+1 \rangle$, so we have $[x^2]+[1]=0$, 
    which implies that $[x^2] = [-1]$. Thus $[x]$ is the square 
    root of $-1$.

\end{exercise}

%%%%%%%%%%%%%%%%%%%%%%%%%%%%%%%%%%%%%%%%

\begin{exercise}{5.2.8}

    \bigskip
    \textbf{(5.2.8a):}
    \begin{proof}
        We have that $\Phi$ is onto, so for any $s \in S$, there 
        exists some $r \in R$ such that $\Phi(r) = s$. It 
        follows that 
        \[
            s = \Phi(r) = \Phi(1r) = \Phi(1)\Phi(r) = s\Phi(1).    
        \]
        Thus, when $s = 1$, we have that $\Phi(1) = 1$ and the 
        identity in $R$ is mapped to the identity in $S$ by $\Phi$.
    \end{proof}

    \bigskip
    \textbf{(5.2.8b):}
    \begin{proof}
        For any $r \in R$, we have that $\Phi(r^{-1})$ is a 
        multiplicative inverse for $\Phi(r)$ since 
        \[
            \Phi(r)\Phi(r^{-1})=\Phi(rr^{-1})=\Phi(1)=1.    
        \]
    \end{proof}

    \bigskip
    \textbf{(5.2.8c):}
        We have shown that S contains the identity along with 
        a multiplicative inverse. To show commutativity, we have 
        that, for some $s_1,s_2 \in S$ such that $\Phi(r_1)=s_1$ 
        and $\Phi(r_2)=s_2$, 
        \[
            s_1s_2=\Phi(r_1)\Phi(r_2)=\Phi(r_1r_2)=\Phi(r_2r_1)=\Phi(r_2)\Phi(r_1)=s_2s_1.    
        \]
\end{exercise}

%%%%%%%%%%%%%%%%%%%%%%%%%%%%%%%%%%%%%%%%
\newpage
\begin{exercise}{5.2.13}

    \bigskip
    \textbf{(5.2.13a):}
    Note that $0 \in \Phi^{-1}(J)$ since $\Phi(0) = \Phi(0)+\Phi(0)=0$,
    Next, we suppose that $\Phi(r_1)=s_1 \in J, \Phi(r_2)=s_2\in J$.
    It follows that 
    \[
        \Phi(r_1+r_2)=\Phi(r_1)+\Phi(r_2)=s_1+s_2,  
    \]
    which implies $r_1+r_2 \in \Phi^{-1}(J)$, since $s_1+s_2 \in J$.
    It also follows that
    \[
        \Phi(r_1r_2)=\Phi(r_1)\Phi(r_2)=\Phi(r_1)s_2,    
    \]
    which implies $r_1r_2 \in \Phi^{-1}(J)$, since $\Phi(r_1)s_2 \in J$.

    \bigskip
    \textbf{(5.2.13b):}
    For any ideal $I \in R$, we must show $\Phi(I)$ is an ideal in $S$.
    We know that $0 \in \Phi(I)$, so it suffices to show closure 
    under addition and multiplication. Let $s_1,s_2 \in \Phi(I)$ such 
    that $\Phi(r_1)=s_1,\Phi(r_2)=s_2$ for some $r_1,r_2 \in I$. 
    We have that 
    \[
        r_1+r_2 \in I, \Phi(r_1+r_2)=\Phi(r_1)+\Phi(r_2)=s_1+s_2 \Rightarrow s_1+s_2 \in \Phi(I).    
    \]
    Similarly, we have that 
    \[
        \Phi(r_1r_2)=\Phi(r_1)\Phi(r_2)=s_1s_2 \Rightarrow s_1s_2 \in \Phi(I)    
    \]
    since $r_1r_2 \in I$.

    Since we know that $\Phi$ is bijective, we have that 
    \[
        \Phi \circ \Phi^{-1}(J)=J, \Phi^{-1}\circ\Phi(I)=I 
    \]
    for ideals $I \in R$ and $J \in S$.

\end{exercise}

%%%%%%%%%%%%%%%%%%%%%%%%%%%%%%%%%%%%%%%%

\begin{exercise}{5.2.16}

    \bigskip
    \textbf{(5.2.16a):}
    $0 \in \ker(\Phi)$ by Exercise 5.2.13a. Assume $r_1,r_2 \in \ker(\Phi)$.
    Then $\Phi(r_1+r_2) = \Phi(r_1)+\Phi(r_2)=0+0$, so $r_1+r_2 \in \ker(\Phi)$.
    We also have that, for some $r \in k[x_1,\ldots,x_n]$,
    $\Phi(rr_1)=\Phi(r)\Phi(r_1)=\Phi(r)\cdot0=0$, so $r\cdot r_1 \in \ker(\Phi)$.
    Therefore, $\ker(\Phi)$ is an ideal.

    \bigskip
    \textbf{(5.2.16b):}
    Assume that $r\equiv r' \mod \ker(\Phi)$ such that $[r]=[r']$ 
    with $r-r'\in \ker(\Phi)$. Since $\Phi(r)-\Phi(r')=\Phi(r-r')=0$,
    it follows that 
    \[
        v([r])=\Phi(r)=\Phi(r')=v([r']).    
    \]
    Therefore, the mapping $v$ is well-defined.

    \bigskip
    \textbf{(5.2.16c):}
    Let $[r_1],[r_2] \in k[x_1,\ldots,x_n]/\ker(\Phi)$. We then have 
    that 
    \begin{align*}
        &v([r_1]+[r_2])  = v([r_1+r_2])=\Phi(r_1+r_2)=\Phi(r_1)+\Phi(r_2)=v([r_1])+v([r_2]),\\
        &v([r_1][r_2])   = v([r_1r_2]) = \Phi(r_1r_2)=\Phi(r_1)\Phi(r_2)=v([r_1])v([r_2]),\\
        &v([1])          = \Phi(1)=1    && \text{by Exercies 5.2.8a}
    \end{align*}

    \bigskip
    \textbf{(5.2.16d):}
    Let $[r_1],[r_2] \in k[x_1,\ldots,x_n]/\ker(\Phi)$ and assume 
    that $v([r_1])=v([r_2])$. It follows that $\Phi(r_1)=\Phi(r_2)$, 
    which implies that $\Phi(r_1-r_2)=0$, i.e., $r_1-r_2 \in \ker(\Phi)$.
    Therefore, $[r_1]=[r_2]$. Thus $v$ is one-to-one, and $v$ 
    being onto follows from the fact that $v([r])=\Phi(r)$ and $\Phi$ 
    is an onto mapping.

\end{exercise}

%%%%%%%%%%%%%%%%%%%%%%%%%%%%%%%%%%%%%%%%

%---------------------------------
% Don't change anything below here
%---------------------------------

\end{document}