\documentclass[12pt,oneside]{article}

% This package simply sets the margins to be 1 inch.
\usepackage[margin=1in]{geometry}

% These packages include nice commands from AMS-LaTeX
\usepackage{amssymb,amsmath,amsthm}


% Define an environment for exercises.
\newenvironment{exercise}[1]{\vspace{.1in}\noindent\textbf{Exercise #1 \hspace{.05em}}}{}

% define shortcut commands for commonly used symbols
\newcommand{\R}{\mathbb{R}}
\newcommand{\C}{\mathbb{C}}
\newcommand{\Z}{\mathbb{Z}}
\newcommand{\Q}{\mathbb{Q}}
\newcommand{\N}{\mathbb{N}}


%%%%%%%%%%%%%%%%%%%%%%%%%%%%%%%%%%%%%%%%%%

\begin{document}

% If you use Overleaf, the name of the project will be determined by
% what you enter as the document title.
\title{Math Homework Template}

\begin{flushright}
\textsc{Marcelo Leszynski}  \\
Rootfinding Research Prerequisites \\
05/10/20
\end{flushright}

\begin{center}
\textsf{Section 1.1 } \\
\textsf{Exercises: 1, 2, 5, 6}
\end{center}

%%%%%%%%%%%%%%%%%%%%%%%%%%%%%%%%%%%%%%%%

\begin{exercise}{1.1.1}
    $F_2$ is a field since it meets the following criteria (not checking the 
    associative and distributive properties):

    \bigskip
    \textbf{Additive Identity:} $0$ satisfies the additive identity property since 
    $1 + 0 = 1$ and $0 + 0=0$.

    \bigskip
    \textbf{Multiplicative Identity:} $1$ satisfies the multiplicative identity property 
    since $1 \cdot 0 = 0$ and $1 \cdot 1 = 1$.

    \bigskip
    \textbf{Additive Inverse:} $0$ satisfies the additive inverse property for $0$ since 
    $0 + 0 = 0$. Likewise, $1$ satisfies the additive inverse property for $1$ 
    since $1 + 1 = 0$.

    \bigskip
    \textbf{Multiplicative Inverse:} The only nonzero element in $F_2$ is $1$, which has 
    a multiplicative inverse of $1$ since $1 \cdot 1 = 1$. 

    \bigskip
    Therefore, $F_2$ is a field.
\end{exercise}

%%%%%%%%%%%%%%%%%%%%%%%%%%%%%%%%%%%%%%%%
\newpage
\begin{exercise}{1.1.2}

    \bigskip
    \textbf{(a)}
    Notice that, for all $f \in F_2$, we have that $f^2 + f = f + f = 0$.
    thus, given the polynomial $g(x,y) = x^2y + y^2x \in F_2[x,y]$, 
    we have that 
    \begin{align*}
        g(x,y)  &= x^2y + y^2x \\
                &= x^2y + xy + y^2x + xy && \text{adding zero}\\
                &= (x^2 + x)y + (y^2 + y)x\\
                &= 0(y) + 0(x)\\
                &= 0 + 0 = 0
    \end{align*}
    as desired. This does not contradict proposition 5 because, in this 
    instance, the field $k = F_2$ is not infinite.

    \bigskip
    \textbf{(b)}
    Define $g(x,y,z) = x^2yz + xy^2z + xyz^2 + xyz$. Then, similarly to 1.1.2a, 
    we have that 
    \begin{align*}
        g(x,y,z)    &= x^2yz + xy^2z + xyz^2 + xyz\\
                    &= x^2yz + xy^2z + xyz^2 + xyz + 4xyz && \text{adding zero}\\
                    &= (x^2yz + xyz) + (xy^2z + xyz) + (xyz^2 + xyz) + (xyz + xyz) && \text{regrouping}\\
                    &= (x^2 + x)yz + (y^2 + y)xz + (z^2 + z)xy + (xyz + xyz)\\
                    &= 0(yz) + 0(xz) + 0(xy) + 0\\
                    &= 0 + 0 + 0 + 0 = 0
    \end{align*}
    as desired.

    \bigskip
    \textbf{(c)}
    Continuing our pattern from parts (a) and (b), we can construct 
    $g(x_1, \ldots, x_n) \in F_2[x_1, \ldots, x_n]$ as follows:
    \[
        g(x_1, \ldots, x_n) = (x_1^2 + x_1)x_2 \ldots x_n + (x_2^2 + x_2)x_1\cdot x_3 \ldots x_n + \ldots + (x_n^2 + x_n)x_1\ldots x_{n-1}
    \]

\end{exercise}

%%%%%%%%%%%%%%%%%%%%%%%%%%%%%%%%%%%%%%%%
\newpage
\begin{exercise}{1.1.5}

    \bigskip
    \textbf{(a)}
    $f(x,y,z) = (y^2z)x^5 - (y^3)x^4 + (z)x^2 + (y + 2)x + (y^5 - y^3z - 5z + 3)$
    
    \bigskip
    \textbf{(b)}
    $f(x,y,z) = y^5 + (-x^4-z)y^3 + (x^5z)y^2 + (x)y + (2x - 5z + 3)$

    \bigskip
    \textbf{(c)}
    $f(x,y,z) = (x^5y^2 + x^2 - y^3 - 5)z + (-x^4y^3 + y^5 + xy + 2x + 3)$

\end{exercise}

%%%%%%%%%%%%%%%%%%%%%%%%%%%%%%%%%%%%%%%%

\begin{exercise}{1.1.6}

    \bigskip
    \textbf{(a)}
    \begin{proof}
        Let $f \in \C[x_1,\ldots, x_n]$ and assume that $f$ vanishes at every 
        point of $\Z^n$. We wish to show that $f$ is the zero polynomial and 
        proceed by induction on $n$.

        \textit{Base case}: Suppose that $n = 1$. Then, since $f(a) = 0$ for 
        all $a \in \Z$, it is obvious that there are an infinite number of 
        roots, which can only occur when $f = 0$ is the zero polynomial.

        \textit{Inductive step}: Assuming the base case and letting $n > 1$,
        we rewrite $f$ as 
        \[
            f = \sum_{i=0}^Ng_i(x_1,\ldots,x_{n-1})x_n^i
        \]
        by collecting powers of $x_n$. Note that $g_i(a_1,\ldots,a_{n-1}) = 0$
        for all $i$. Thus, by induction, we have that $g_i \in \C[x_1,\ldots,x_{n-1}]$ 
        gives the zero function, which implies that $f$ is the zero polynomial in
        $\C[x_1,\ldots,x_n]$.
    \end{proof}

    \bigskip
    \textbf{(b)}
    Let $f \in \C[x_1\ldots x_n]$ and let $M$ be the largest power of any 
    variable that appears in $f$. Let $\Z_{M+1}^n$ be the set of points of 
    $\Z^n$, all coordinates of which lie between $1$ and $M+1$ inclusive. 

    \begin{proof}
        We wish to show that if $f$ vanishes at all points of $\Z_{M+1}^n$, then 
        $f$ is the zero polynomial. We begin by assuming that $f$ vanishes 
        at all points of $\Z_{M+1}^n$ and work by induction on $n$. 

        \textit{Base case}: Suppose that $n = 1$. Then $f$ is a polynomial 
        in one variable with a maximum degree of $M$. If we have that 
        $f(1) = f(2) = \ldots = f(M) = f(M + 1) = 0$, we can conclude that 
        $f$ must have at least $M + 1$ distinct roots. Since the number of 
        distinct roots is greater than the degree of the polynomial, it must 
        be that the polynomial is the zero polynomial $f = 0$.

        \textit{Inductive step}: Assuming the base case and letting $n > 1$, 
        we rewrite $f$ as
        \[
            f = \sum_{i=0}^M g_i(x_1, \ldots, x_{n-1})x_n^i
        \]
        with $g_i \in \C[x_1, \ldots, x_{n-1}]$. We know then that, for any 
        $(a_1,\ldots, a_{n-1}) \in \Z_{M+1}^{n-1}$, we have 
        \[
            f(a_1,\ldots,a_{n-1}, x_n) = \sum_{i=0}^Mg_i(a_1,\ldots,a_n-1)x_n^i.
        \]
        We see that $g_i(a_1,\ldots,a_{n-1}) = 0$ for all $i = 0,\ldots, M$. 
        Now we have that $g_i$ is the zero polynomial by induction, which 
        implies that $f$ is also the zero polynomial.
    \end{proof}
\end{exercise}

%%%%%%%%%%%%%%%%%%%%%%%%%%%%%%%%%%%%%%%%


%---------------------------------
% Don't change anything below here
%---------------------------------


\end{document}