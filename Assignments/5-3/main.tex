\documentclass[12pt,oneside]{article}

% This package simply sets the margins to be 1 inch.
\usepackage[margin=1in]{geometry}

% These packages include nice commands from AMS-LaTeX
\usepackage{amssymb,amsmath,amsthm,graphicx}

% Define an environment for exercises.
\newenvironment{exercise}[1]{\vspace{.1in}\noindent\textbf{Exercise #1 \hspace{.05em}}}{}

% define shortcut commands for commonly used symbols
\newcommand{\R}{\mathbb{R}}
\newcommand{\C}{\mathbb{C}}
\newcommand{\Z}{\mathbb{Z}}
\newcommand{\Q}{\mathbb{Q}} 
\newcommand{\N}{\mathbb{N}} 

%%%%%%%%%%%%%%%%%%%%%%%%%%%%%%%%%%%%%%%%%%

\begin{document}

% If you use Overleaf, the name of the project will be determined by
% what you enter as the document title.
\title{Math Homework Template}

\begin{flushright}
\textsc{Marcelo Leszynski}  \\
Rootfinding Research Prerequisites\\
10/07/20
\end{flushright}

\begin{center}
\textsf{Assignment 5.3 } \\
\textsf{Exercises: 2, 3, 4, 5, 7, 11 }
\end{center}

%%%%%%%%%%%%%%%%%%%%%%%%%%%%%%%%%%%%%%%%

\begin{exercise}{5.3.2}
    \begin{proof}
        We have that $\overline{f\cdot g}^G=\overline{\overline{f}^G\overline{g}^G}^G$,
        so the remainders of $f$ and $g$ do not need to be computed 
        separately. Instead, for $[f] \cdot [g] \in k[x_1,\ldots,x_n]/I$,
        $\overline{f\cdot g}^G$ can be computed.
    \end{proof}
\end{exercise}

%%%%%%%%%%%%%%%%%%%%%%%%%%%%%%%%%%%%%%%%

\begin{exercise}{5.3.3}
    We can compute a Grobner basis of $I$ with lex order $x>y>z$ 
    as 
    \[
        x^2-z^{55},xz^6-z^{64}, y^3-z^{54},yz^6-z^{24},z^{67}-z^6,
    \]
    and a basis with grlex order $x>y>z$ as 
    \[
        x^9-x^2y^2z^4,x^2y^7-x^2z^4,y^9-x^7,x^7y-x^2z^3,x^4y^4-x^2z^5,x^5z-y^6,z^6-x^4y,x^3z^2-y^3,y^3z-x^2.    
    \]
\end{exercise}

%%%%%%%%%%%%%%%%%%%%%%%%%%%%%%%%%%%%%%%%

\begin{exercise}{5.3.4}
    \begin{proof}
        The case where $c=0$ is trivial. Let $c \in k$ be nonzero, 
        and let $r=\overline{c\cdot f}^G$ be the unique remainder such 
        that $cf=q+r$ for some $q \in I$. It follows that $r$ is a 
        $k$-linear combination of the monomials in the complement of 
        $\langle$LT$(I)\rangle$. We then divide by $c$ to get $f=\frac{q}{c}+\frac{r}{c}$.
        We have $\frac{q}{c} \in I$ since $c \in k$, and the monomials 
        in $\frac{r}{c}$ are identical as the monomials in $r$. By 
        Proposition 5.3.1, $\frac{r}{c}=\overline{f}^G$, so $\overline{c\cdot f}^G = c \cdot \overline{f}^G$.
    \end{proof}
\end{exercise}

%%%%%%%%%%%%%%%%%%%%%%%%%%%%%%%%%%%%%%%%

\begin{exercise}{5.3.5}

    \bigskip
    \textbf{(5.3.5a):}
    Let $I$ be defined as in the text, and let $f \in \R[x,y]/I$ 
    be arbitrary. Using Proposition 5.3.4, we have that 
    \[
        [f] = c_1[1]+c_2[x]+c_3[y]+c_4[y^2]    
    \]
    is a unique representation of $f$ with $c_i \in \R$. Note 
    that the mapping $\phi([f])=(c_1,c_2,c_3,c_4)^T$ is injective 
    between $\R[x,y]/I$ and $\R^4.$

    Let $c \in \R$ and let 
    \[
        [g]=d_1[1]+d_2[x]+d_3[y]+d_4[y^2] \in \R[x,y]/I.    
    \]
    Then 
    \begin{align*}
        \phi([f]+[g])   &= \phi((c_1+d_1)[1]+(c_2+d_2)[x]+(c_3+d_3)[y]+(c_4+d_4)[y^2])\\
                        &=(c_1+d_1,c_2+d_2,c_3+d_3,c_4+d_4)^T=(c_1,c_2,c_3,c_4)^T+(d_1,d_2,d_3,d_4)^T\\
                        &=\phi([f])+\phi([g]),
    \end{align*}
    and 
    \begin{align*}
        \phi(c[f])  &= \phi(cc_1[1]+cc_2[x]+cc_3[y]+cc_4[y^2])=(cc_1,cc_2,cc_3,cc_4)^T=c(c_1,c_2,c_3,c_4)^T\\
                    &=c\phi([f]).
    \end{align*}
    Thus, by definition, $\phi$ is a linear map, and we conclude that 
    $\R[x,y]/I \simeq \R^4$ by $\phi$.

    \bigskip
    \textbf{(5.3.5b):}
    Using the given Grobner basis, we express each product as a 
    linear combination of $\{[1],[x],[y],[y^2]\}$ as follows:
    \begin{align*}
        x\cdot x    &=1\cdot(x^2+y-1)-y+1 \Rightarrow [x]\cdot[x]=-[y]+[1],\\
        x\cdot y    &=1\cdot(xy-2y^2+2y)+2y^2-2y \Rightarrow [x]\cdot[y]=2[y^2]-2[y],\\
        x\cdot y^2  &=y(xy-2y^2+2y)+2(y^3-(7/4)y^2+(3/4)y)+(3/2)y^2-(3/2)y\\
                    & \Rightarrow [x]\cdot[y^2]=(3/2)[y^2]-(3/2)[y],\\
        y\cdot y^2  &=1\cdot(y^3-(7/4)y^2+(3/4)y)+(7/4)y^2-(3/4)y \Rightarrow [y]\cdot[y^2]=(7/4)[y^2]-(3/4)[y]\\
        y^2 \cdot y^2 &= (y+7/4)(y^3-(7/4)y^2+(3/4)y)+(37/16)y^2-(21/16)y\\
                    & \Rightarrow [y^2]\cdot[y^2]=(37/16)[y^2]-(21/16)[y].
    \end{align*}

    \bigskip
    \textbf{(5.3.5c):}
    $\R[x,y]/I$ is not a field since 
    \[
        ([x]-2[y]+2)\cdot[y] = [xy-2y^2+2y]=0
    \]
    so $[y]$ is a nonzero divisor in $\R[x,y]$.

    \bigskip
    \textbf{(5.3.5d):}
    We factor and solve the last equation in the Grobner basis $G$, 
    which results in 
    \[
        \frac{1}{4}\cdot y(4y^2-7y+3)=\frac{1}{4}\cdot y(y-1)(4y-3)=0     
    \]
    so $y = 0,1,\frac{4}{3}$. Back-substituting into the first two 
    equations of $G$ and have that 
    \[
        \mathbf{V}(I)=\{(0,1),(0,-1),(1,0),(\frac{3}{4},-\frac{1}{2})\}.
    \]

    \bigskip
    \textbf{(5.3.5e):}
    Part $(ii)$ of Proposition 7 gives that $\mathbf{V}(I)$ has 
    at most $6$ points. Part $(i)$ of the proposition gives a 
    better bound since $\R[x,y]/I$ is of dimension $4$ so it 
    gives a bound of $4$.

\end{exercise}

%%%%%%%%%%%%%%%%%%%%%%%%%%%%%%%%%%%%%%%%

\begin{exercise}{5.3.7}

    \bigskip
    \textbf{(5.3.7a):}
    If $\{x^\alpha \vert x^\alpha \not \in \langle$LT$(I)\rangle\}$ contains 
    $d$ elements $x^{\alpha_1},\ldots,c^{\alpha_2}$, Proposition 4 
    gives that $k[x_1,\ldots,x_n]/I$ is a $k$-vector space that 
    is isomorphic to $S =$Span$(x^{alpha_1},\ldots,x^{alpha_2})$. 
    Since each $\alpha_i$ is distinct, it follows that all $x^{\alpha_1}$ 
    are linearly independent, so $S$ has dimension $d$. Because $S$ 
    is isomorphic to $k[x_1,\ldots,x_n]/I$, it follows that $k[x_1,\ldots,x_n]/I$ 
    is also of dimension $d$.

    \bigskip
    \textbf{(5.3.7b):}
    We conclude that the number of monomials in the complement of $\langle$LT$(I)\rangle$ 
    is independent of the choice of monomial order when that number 
    is finite since Proposition 4 and Exercise 5.3.7a are independent 
    of the choice of monomial order.

\end{exercise}

%%%%%%%%%%%%%%%%%%%%%%%%%%%%%%%%%%%%%%%%

\begin{exercise}{5.3.11}

    \bigskip
    \textbf{(5.3.11a):}
    We compute a Grobner basis for $I$ with lex order $x>y>z$ as 
    \[
        x_z+3z^2+z^4-z^5,y+3z+6z^2+2z^3+z^4-2z^5, -z-3z^2-3z^3-z^4+z^6.
    \]
    Using Theorem 5.3.6, we give the table of monomials $m_d$ of 
    total degree $\leq d$ that are not in $\langle$LT$(I)\rangle$ 
    with $1\leq d\leq 10$:

    \begin{tabular}{l|l|l}
        $d$ & $m_d$ & monomial set $M_d$\\
        \hline
        1&2&$\{1,z\}$\\
        2&3&$\{1,z,z^2\}$\\
        3&4&$\{1,z,z^2,z^3\}$\\
        4&5&$\{1,z,z^2,z^3,z^4\}$\\
        $\geq5$&6&$\{1,z,z^2,z^3,z^4,z^5\}$
    \end{tabular}

    \bigskip
    \textbf{(5.3.11b):}
    Similarly to part (a), we compute a Grobner basis 
    \[
        x-y^2+z^2,y^4-2y^2z^2+y+z^4    
    \]
    and form a table of values:

    \begin{tabular}{l|l|l}
        $d$&$m_d$&monomial set $M_d$\\
        \hline
        1&3&$\{1,y,z\}$\\
        2&6&$\{1,y,z,yz,y^2,z^2\}$\\
        3&10&$\{1,y,z,yz,y^2,z^2,y^2z,yz^2,y^3,z^3\}$\\
        4&14&$\{1,y,z,yz,y^2,z^2,y^2z,yz^2,y^3,z^3,yz^3,y^2z^2,y^3z,z^4\}$\\
        5&6&$\{1,y,z,yz,y^2,z^2,y^2z,yz^2,y^3,z^3,yz^3,y^2z^2,y^3z,z^4,y^3z^2,y^2z^3,yz^4,z^5\}$\\
        6&22&$\{1,y,z,yz,y^2,z^2,y^2z,yz^2,y^3,z^3,yz^3,y^2z^2,y^3z,z^4,y^3z^2,y^2z^3,yz^4,z^5y^3z^3,y^2z^4,yz^5,z^6\}$\\
        7&26&$\{1,y,z,yz,y^2,z^2,y^2z,yz^2,y^3,z^3,yz^3,y^2z^2,y^3z,z^4,y^3z^2,y^2z^3,yz^4,z^5y^3z^3,y^2z^4,yz^5,z^6,$\\
        & &$y^3z^4,y^2z^5,yz^6,z^7\}$\\
        $\geq8$&$4d-2$&$M_d=M_{d-1}\cup \{y^3z^{d-3},y^2z^{d-2},yz^{d-1},z^d\}$
    \end{tabular}

    \bigskip
    \textbf{(5.3.11c):}
    This is given ad $H(d)=4d-2$ by Exercise 5.3.11b, so $H(d)$ is 
    linear.

    \bigskip
    \textbf{(6.3.11d):}
    We compute a Grobner basis $x^2+y$ and form a table of values:

    \begin{tabular}{l|l|l}
        $d$&$m_d$&monomial set $M_d$\\
        \hline
        1&4&$\{1,x,y,z\}$\\
        2&9&$\{1,x,y,z,xy,xz,yz,y^2,z^2\}$\\
        3&16&$\{1,x,y,z,xy,xz,yz,y^2,z^2,xy^2,xyz,xz^2,y^3,y^2z,yz^2,z^3\}$\\
        $\geq4$&$(d+1)^2$&$M_d=m_{d-1}\cup \{xy^iz^k \vert i+k=d-1\}\cup \{y^mz^n\vert m+n=d\}$
    \end{tabular}

    \bigskip
    \textbf{(5.3.11e):}
    It appears that dimension corresponds to the degree of the 
    function $H(d)$, or in other words, each equation in the 
    definition of a variety imposes an additional constraint that 
    reduces the possible dimension by one.

\end{exercise}

%%%%%%%%%%%%%%%%%%%%%%%%%%%%%%%%%%%%%%%%

%---------------------------------
% Don't change anything below here
%---------------------------------

\end{document}